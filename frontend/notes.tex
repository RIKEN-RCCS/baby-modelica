%% notes.tex -*-Coding: us-ascii-unix; Mode: Fundamental;-*-

%% Memo on Baby-Modelica

\documentclass[10pt,b5paper]{article}
\usepackage[b5paper,margin=1.5cm]{geometry}

\title{\bf{}Baby-Modelica Implementation Notes}
\author{Copyright (C) 2018-2021 RIKEN R-CCS}
\date{Baby-Modelica 3.4.0, 2020-01-29}

\usepackage[hidelinks]{hyperref}
\usepackage{listings}
\usepackage{amssymb} %% (for a checkmark, mathbb)

\makeatletter
%%\renewcommand*\l@section{\@dottedtocline{1}{1.5em}{2.3em}}
%%\renewcommand*\l@subsection{\@dottedtocline{1}{3em}{2em}}
\makeatother

\setcounter{tocdepth}{2}

%%\usepackage[linkbordercolor={0 0 1}]{hyperref}

\lstset{basicstyle=\ttfamily,frame=leftline,columns=flexible,keepspaces=true}
%%\lstset{basicstyle=\ttfamily\small,numbers=left,firstnumber=0,frame=leftline}

%% Numbered quotations.

\newcounter{quoteno}
\newcounter{ruleno}
\def\tombstone{\rule[-.2ex]{.3em}{1.8ex}}

\def\quotelead{\refstepcounter{quoteno}{\bf{}Quote~\arabic{quoteno}}.}
\newenvironment{qquote}{\begin{quote}\quotelead}%
{\end{quote}\ignorespacesafterend}

\newenvironment{specquote}[1]{\begin{qquote} (\specref{#1}\/) }%
{\end{qquote}}

\newenvironment{widequote}{%
\list{}{\leftmargin=0mm\rightmargin=\leftmargin}%
\item\relax}{\endlist}

\def\rulelead{\refstepcounter{ruleno}{\bf{}Rule~\arabic{ruleno}}.}
\newenvironment{QRULE}{\begin{widequote}\rulelead}%
{\unskip\nobreakspace\tombstone\end{widequote}\ignorespacesafterend}

\def\factlead{\refstepcounter{ruleno}{\bf{}Fact~\arabic{ruleno}}.}
\newenvironment{QFACT}{\begin{widequote}\factlead}%
{\unskip\nobreakspace\tombstone\end{widequote}\ignorespacesafterend}

\def\bugslead{{\bf{}Bugs}.}
\newenvironment{qbugs}{\begin{quote}\bugslead}%
{\end{quote}\ignorespacesafterend}

\def\quoteref#1{Qoute~\ref{#1}}
\def\ruleref#1{Rule~\ref{#1}}
\def\factref#1{Fact~\ref{#1}}

%% Section titles of the_specification.

\def\specrefx#1#2{Specification~#1\ifx\relax#2\relax{}\else~{\it{}#2}\fi}
\def\specref#1{\specrefx{#1}{\csname spec#1\endcsname}}
\def\defsection#1#2{\expandafter\def\csname spec#1\endcsname{#2}}

%% notes-spectoc.tex -*-Coding: us-ascii-unix;-*-

%% Section titles of the specification.

\let\deftoc=\defsection
\deftoc{2.3.3}{Modelica Keywords}
\deftoc{3.2}{Operator Precedence and Associativity}
\deftoc{3.5}{Equality, Relational, and Logical Operators}
\deftoc{3.6.7}{Built-in Variable time}
\deftoc{3.7}{Built-in Intrinsic Operators with Function Syntax}
\deftoc{3.7.1}{Numeric Functions and Conversion Functions}
\deftoc{3.7.1.1}{Event Triggering Mathematical Functions}
\deftoc{3.7.1.2}{Built-in Mathematical Functions and External Built-in Functions}
\deftoc{3.7.2}{Derivative and Special Purpose Operators with Function Syntax}
\deftoc{3.7.3}{Event-Related Operators with Function Syntax}
\deftoc{3.8}{Variability of Expressions}
\deftoc{3.8.1}{Constant Expressions}
\deftoc{3.8.2}{Parameter Expressions}
\deftoc{3.8.3}{Discrete-Time Expressions}
\deftoc{4.1}{Access Control -- Public and Protected Elements}
\deftoc{4.4.2}{Component Declaration Static Semantics}
\deftoc{4.4.4}{Component Variability Prefixes discrete, parameter, constant}
\deftoc{4.4.5}{Conditional Component Declaration}
\deftoc{4.5.1}{Short Class Definitions}
\deftoc{4.5.2}{Restriction on combining base-classes and other elements}
\deftoc{4.6}{Specialized Classes}
\deftoc{4.8}{Predefined Types and Classes}
\deftoc{4.8.5}{Enumeration Types}
\deftoc{4.8.5.4}{Unspecified enumeration}
\deftoc{4.8.8}{Other Predefined Types}
\deftoc{5.1}{Flattening Context}
\deftoc{5.2}{Enclosing Classes}
\deftoc{5.3.1}{Simple Name Lookup}
\deftoc{5.3.2}{Composite Name Lookup}
\deftoc{5.4}{Instance Hierarchy Name Lookup of Inner Declarations}
\deftoc{5.4.1}{Example of Field Functions using Inner/Outer}
\deftoc{5.5}{Simultaneous Inner/Outer Declarations}
\deftoc{5.6}{Flattening Process}
\deftoc{5.6.1}{Instantiation}
\deftoc{5.6.1.1}{The Class Tree}
\deftoc{5.6.1.2}{The Instance Tree}
\deftoc{5.6.1.4}{Steps of Instantiation}
\deftoc{5.6.2}{Generation of the flat equation system}
\deftoc{6.3}{Interface Compatibility or Subtyping}
\deftoc{6.4}{Plug-Compatibility or Restricted Subtyping}
\deftoc{6.5}{Function-Compatibility or Function-Subtyping for Functions}
\deftoc{7.1}{Inheritance -- Extends Clause}
\deftoc{7.1.3}{Restrictions on the Kind of Base Class}
\deftoc{7.1.4}{Restrictions on Base Classes and Constraining Types to be Transitively Non-Replaceable}
\deftoc{7.2}{Modifications}
\deftoc{7.2.2}{Modification Environment}
\deftoc{7.2.5}{Modifiers for Array Elements}
\deftoc{7.2.6}{Final Element Modification Prevention}
\deftoc{7.3}{Redeclaration}
\deftoc{7.3.1}{The class extends Redeclaration Mechanism}
\deftoc{7.3.3}{Restrictions on Redeclarations}
\deftoc{7.3.4}{Annotation Choices for Suggested Redeclarations and Modifications}
\deftoc{8.3.1}{Simple Equality Equations}
\deftoc{8.3.2}{For-Equations - Repetitive Equation Structures}
\deftoc{8.3.7}{assert}
\deftoc{8.3.8}{terminate}
\deftoc{8.6}{Initialization, initial equation, and initial algorithm}
\deftoc{9.3}{Restrictions of Connections and Connectors}
\deftoc{9.4}{Equation Operators for Overconstrained Connection-Based Equation Systems}
\deftoc{9.4.1}{Overconstrained Equation Operators for Connection}
\deftoc{10}{Arrays}
\deftoc{10.1}{Array Declarations}
\deftoc{10.1.1}{Array Dimension Lower and Upper Index Bounds}
\deftoc{10.3}{Built-in Array Functions}
\deftoc{10.3.1}{Array Dimension and Size Functions}
\deftoc{10.3.2}{Dimensionality Conversion Functions}
\deftoc{10.3.3}{Specialized Array Constructor Functions}
\deftoc{10.3.4}{Reduction Functions and Operators}
\deftoc{10.3.5}{Matrix and Vector Algebra Functions}
\deftoc{10.4}{Vector, Matrix and Array Constructors}
\deftoc{10.4.1}{Array Constructor with Iterators}
\deftoc{10.4.2}{Array Concatenation}
\deftoc{10.4.3}{Vector Construction}
\deftoc{10.5}{Array Indexing}
\deftoc{10.5.2}{Indexing with end}
\deftoc{10.6}{Scalar, Vector, Matrix, and Array Operator Functions}
\deftoc{10.6.2}{Array Element-wise Addition, Subtraction, and String Concatenation}
\deftoc{10.6.3}{Array Element-wise Multiplication}
\deftoc{10.6.4}{Matrix and Vector Multiplication of Numeric Arrays}
\deftoc{10.6.5}{Division of Scalars or Numeric Arrays by Numeric Scalars}
\deftoc{10.6.6}{Array Element-wise Division}
\deftoc{10.6.7}{Exponentiation of Scalars of Numeric Elements}
\deftoc{10.6.8}{Scalar Exponentiation of Square Matrices of Numeric Elements}
\deftoc{10.6.10}{Relational Operators}
\deftoc{10.6.11}{Boolean Operators}
\deftoc{10.7}{Empty Arrays}
\deftoc{11.2.1.1}{Assignments from Called Functions with Multiple Results}
\deftoc{11.2.2.3}{Nested For-Loops and Reduction Expressions with Multiple Iterators}
\deftoc{12.2}{Function as a Specialized Class}
\deftoc{12.3}{Pure Modelica Functions}
\deftoc{12.4.6}{Scalar Functions Applied to Array Arguments}
\deftoc{12.7.1}{Using the Derivative Annotation}
\deftoc{12.8}{Declaring Inverses of Functions}
\deftoc{12.9.1.2}{Arrays}
\deftoc{12.9.4}{Annotations for External Libraries and Include Files}
\deftoc{13.2.1}{Importing Definitions from a Package}
\deftoc{13.2.1.1}{Lookup of Imported Names}
\deftoc{13.2.2}{Mapping Package/Class Structures to a Hierarchical File System}
\deftoc{13.2.4.1}{Example of Searching MODELICAPATH}
\deftoc{14}{Overloaded Operators}
\deftoc{14.2}{Overloaded Constructors}
\deftoc{14.3}{Overloaded String Conversions}
\deftoc{14.4}{Overloaded Binary Operations}
\deftoc{14.5}{Overloaded Unary Operations}
\deftoc{14.6}{Example of Overloading for Complex Numbers}
\deftoc{16.4}{Discrete States}
\deftoc{16.5.1}{Base-clock conversion operators}
\deftoc{16.5.2}{Sub-clock conversion operators}
\deftoc{16.10}{Other Operators}
\deftoc{17.1}{Transitions}
\deftoc{18.3}{Annotations for Code Generation}
\deftoc{18.4}{Annotations for Simulation Experiments}
\deftoc{18.5}{Annotation for single use of class}
\deftoc{18.7}{Annotations for the Graphical User Interface}
\deftoc{18.8.2}{Version Handling}


%% "_" as a space (to let key words searchable).

\catcode`\_=\active
\def_{\relax\ifmmode\sb\else\ \fi}

\let\PARAGRAPH=\paragraph

\if 0
\def\xxtt{\small\tt}
\makeatletter
\def\spacex{\leavevmode\penalty10000\ }
\def\delparx#1{\ifx#1\par\leavevmode\else\ifx#1\spacex\leavevmode\else#1\fi\fi%
\parskip=0pt}
\def\lbracex{\char`\{}\def\rbracex{\char`\}}
\def\docspecials{\do\ \do\$\do\&\do\#\do\^\do\^^K\do\_\do\^^A\do\%\do\~}
\obeyspaces\gdef\xtt{\xxtt\catcode``=13\@noligs\let\do\@makeother\docspecials%
\let\parx\par\def\par{\hbox{}\parx}\obeylines%
\frenchspacing\catcode`\ =\active\let =\spacex%
\let\{=\lbracex\let\}=\rbracex\delparx}
\def\endxtt{}
\makeatother
\fi

\begin{document}

\maketitle

\tableofcontents

%%\section*{Preface}
%%\addcontentsline{toc}{chapter}{Preface}

%%%%%%%%%%%%%%%%%%%%%%%%%%%%%%%%%%%%%%%%%%%%%%%%%%%%%%%%%%%%%%%%

\section{Introduction}

%%%%%%%%%%%%%%%%

\subsection{Baby-Modelica}

Baby-Modelica is a simple frontend (parser+flattener) for the Modelica
language specification 3.4.  It parses the source code then performs
simple syntactic transformations, and optionally dumps a flat model.
Flattening a model is an essential part of Modelica, and it resolves
all the features as an object-oriented language.  Flattening is
indispensable even for simple tools, because parameters in a model are
substituted multiple times by nested parameterization.  A generated
flat model is hopefully to be re-readable by Modelica compilers.

Current status is a pre-zero version, and we are working towards
version zero.  The version number of Baby-Modelica is the last digits,
appended to the version number of the Modelica language specification.

This random memo accompanies the source code.  We apologize if it is
illegible.  This document often refers to the sections of
the_specification.

%%%%%%%%%%%%%%%%

\subsection{Limitations and general remarks}

\begin{itemize}

\item The_implementation is written in SML~'97 (Standard ML).  The
code is developed mainly with Poly/ML (on SunOS~5.11/amd64), but it
may work with SML/NJ or MLton.  Sometimes the code is twisted because
SML does not allow forward references of names.

\item The_implementation uses a modified BYACC parser generator (not
ml-yacc), which can be obtained separately.

\item The file coding is ASCII, although it is defined as UTF-8 in
\specref{13.2.2}.

\item Error checking/reporting is none.  The parser records no source
line number information.  No information is provided for errors in a
model.

\item Constant folding is very weak.  Folding constants at
translation-time is mandatory for array dimensions.

\item The_implementation may be confused by the uses of characters
"{\tt{}.}" and "{\tt{}@}" in names in a model.  It internally uses
them to separate identifiers.

\item The_implementation treats $\mathbf{class}$ as a synonym of
$\mathbf{model}$, that follows the old definition.

\item The_implementation does not implement $\mathbf{pure}$
expressions and errs on the uses of them.

\item It does not parse the {\it{}Synchronous Language\/} and
{\it{}State Machines}.

\item Internal errors (assertions) raise the Match exception.  But,
other errors are often synonyms of the Match exception.

\item The_implementation does not check the representable values of
{Integer}.  Escpecially, {Integer\_inf} may raise the Overflow
exception.  The definition of {ModelicaServices.Integer\_inf} in
MSL~3.2.3 is in 32~bits, and it is no problem for 64~bit SML
implementations.

\item (* Array subscripts ":" to connector references are not
handled. *).

\item (* The code for expanding for-equations and finding implicit
iterator ranges is not tested at all. *).

\item The package Connectors cannot be extended.  It is handled
nominally.

\item (?) Translation-time constants cannot depend on variables in
expandable_connectors.

\end{itemize}

%%%%%%%%%%%%%%%%

\subsection{Useful documents}

\begin{itemize}

\item {\it{}Modelica Language Specification}~\cite{specification34} is
the source of the definitions.  Note that some concepts are defined
not in the text but in the glossary.

\item {\it{}OpenModelica System Documentation}~\cite{openmodelica2014}
explains the implementation of OpenModelica.  See the section
{\it{}Structure}, for Parse and Elaborate/Instantiate/Flatten.

\item {\it{}DSblock: A neutral description of dynamic
systems}~\cite{otter1994} is an old description of structuring
interfaces to solvers.

\item {\it{}Translator to flat Modelica}~\cite{lms2007} describes some
notes about a translator from Modelica subset to flat Modelica, which
is under development as a tool for AMESim.

\item {\it{}SysML--Modelica Transformation}~\cite{omg2012} describes a
translator from SysML to Modelica.  (The overview says bi-directional
but it seems not).  There is an accompanying paper~\cite{paredis2010}.

\end{itemize}

%%%%%%%%%%%%%%%%%%%%%%%%%%%%%%%%%%%%%%%%%%%%%%%%%%%%%%%%%%%%%%%%

\section{Parsing}

{Parsing} is simply done by a BYACC generated parser.

%%%%%%%%%%%%%%%%

\subsection{Code tips}

\begin{itemize}

\item The rule set in {\it{}Modelica Concrete Syntax\/} defined in
the_specification is used essentially verbatim.

\item The keywords {\bf{}end} and {\bf{}initial} appearing in the
{\it{}primary\/} expression of the syntax rules cause conflicts in
BYACC\@.

\end{itemize}

%%%%%%%%%%%%%%%%

\subsection{Parsing of {\bf{}end} and {\bf{}initial}}

The keywords {\bf{}end} and {\bf{}initial} appearing in the
{\it{}primary\/} expression of the syntax rules cause conflicts in
BYACC\@.  {\bf{}end} may appear as a predefined variable as well as a
keyword.  {\bf{}initial} may also appear as a predefined function as
well as a keyword.  For {\bf{}end}, see \specref{10.5.2}.  For
{\bf{}initial}, see \specref{8.6}.

The token {\bf{}end} only appears as a variable in subscript
expressions.  The lexer recognizes {\bf{}end} as an identifier in
subscript expressions, or as a keyword elsewhere.  The parser tells
the lexer to switch the token type of {\bf{}end} as a keyword or an
identifier.  The switching code is placed at the safe points where a
look-ahead of the parser never looks at {\bf{}end}.

The token {\bf{}initial} only appears as a keyword either in
{\bf{}initial equation} or {\bf{}initial algorithm}.  The lexer
handles each sequence of {\bf{}initial~equation} and
{\bf{}initial~algorithm} as a single token.  The lexer judges by a
look-ahead of a token next to {\bf{}initial}.  Otherwise, the lexer
returns an identifier for {\bf{}initial}.

%%%%%%%%%%%%%%%%%%%%%%%%%%%%%%%%%%%%%%%%%%%%%%%%%%%%%%%%%%%%%%%%

\section{Resolving}

{Resolving} handles modifier applications in a model for elaboration.

%%%%%%%%%%%%%%%%

\subsection{Code tips}

\begin{itemize}

\item The_implementation directly handles a short_class_definition as
a class {\it{}refining\/} which represents modifier applications,
although it is defined as an expansion to a class definition with an
extends-clause in \specref{4.5.1}.

\item Mutual extendings are ignored (not errors).

\item The_implementation does not save modified classes (even though
they are named), and repeats the same modifications each time a
package/instance is processed.

%% \item The_implementation refers to a toplevel model by the name
%% "model" (it just looks better than an empty name).

\end{itemize}

%%%%%%%%%%%%%%%%

\subsection{Internal class naming}

Most of the classes do not have names, because a modifier application
creates a new class but it may not have a name unless explicitly
given.  Also, name association may be changed by inner-outer matching
and redeclarations.

Thus, the_implementation refers to a class by a package/instance name,
supplementally paired with a class name.  A pair is used becuase only
main (non-base) classes are named in this way, but names of base
classes are needed to distinguish them.  For example, it may refer to
one base class with a pair
$\mathit{tank}|\mathit{PartialLumpedVessel}$, where $\mathit{tank}$ is
an instance of $\mathit{OpenTank}$ and $\mathit{PartialLumpedVessel}$
is a base of it.

A class being named by a package/instance name is appropriate, because
only classes used are need to be accessed.  Note that all class bodies
(a list of elements) have names, with the exception of a body of an
extends-redeclaration, for which the base class name is used as a name
for it.

%%%%%%%%%%%%%%%%

\subsection{Rewriting import-clauses}

The parser rewrites an import-clause to an explicitly named form to
reduce the number of variations.  This rewriting is performed in the
parser.  For example, it rewrites:
%%%%
\begin{quote}
\begin{lstlisting}[aboveskip=-\baselineskip]
import pkg.n;
==>
import n = pkg.n;

import pkg.{n0,n1,n2};
==>
import n0 = pkg.n0;
import n1 = pkg.n1;
import n2 = pkg.n2;
\end{lstlisting}
\end{quote}

%%%%%%%%%%%%%%%%

\subsection{Handling of modifiers}

\subsubsection*{Modifier applications}

The_syntaxer associates an application of modifiers to a declaration
which is converted to a modification.  Association to a class
definition $C$ is by temporarily creating a syntax element
$\mathit{refine}\,(C, ...)$ with modifiers.

\subsubsection*{Redeclarations}

The_syntaxer handles a redeclaration by replacing a replaceable.
Similarly, it handles an extends-redeclaration by replacing a
replaceable class and making it a base class.

A class redefinition may need be applied to multiple occurrences of a
replaceable in class elements, because the_syntaxer delays merging
elements in a main and base classes.  See
Rule~\ref{multiple-definitions}.

\subsubsection*{Scoped modifiers}

Names and expressions are attached with an environment which records
the scope where they occur, especially for the RHS of modifiers.  This
makes modifiers to be moved inside another class which are associated
to.  The scope is represented by either an instance name or a class
name.  This scoping is necessary to delay the name resolution of
modifiers attached to extends-clauses.  Such modifiers have to be
moved inside a base class before resolving the modifiers, because it
lacks the information of the base classes at that time.

\subsubsection*{A value attribute}

An initializer_modifier to the simple_types and a modifier to the
$\mathit{value}$ attribute are synonymous.  The_implementation handles
them by converting an initializer_modifier to a $\mathit{value}$
modfier.  See Fact~\ref{value-attribute} about a $\mathit{value}$
attribute.

\subsubsection*{Initializer_modifiers}

An initializer_modifier like $x=e$ is converted to modifiers
corresponding to components.  That is, $x=e$ is converted to
$x\,(c_0=e.c_0,c_1=e.c_1,...)$ for each component $c_i$ of the class
$x$.  It does not check $x$ and $e$ have the same set of components;
$x$ can be of fewer components than $e$.  It will make many copy
expressions of $e$.  (?) It will be problematic when the expression
$e$ has effects.

%%%%%%%%%%%%%%%%

\subsection{Modifcations and lookups}

Modifications are shallowly applied to the class, and then lookups of
the class names are performed.  Here, shallow application means that
each modifier is pushed inside the present class and it is attached as
a modifier to a corresponding element.  The process does not
immediately recurse.  Name lookups need to be performed after
modification application, because redeclaration modifications affect
lookups of names of bases classes.  Thus, modifications and lookups
are performed alternatingly.

%%%%%%%%%%%%%%%%

\subsection{Lookups of base_classes}

The resolving of a class of an extends-clause may involve resolving
classes of the parts of a component name.  During the search, many new
resolving processes may be started for intermediate packages.
Resolving processes may interlace (which may visit a class being under
resolving), and the search sees classes with unfinished resolving.
The_implementation keeps track of a process by the three states of
loaded, importing-resolved, and extending-resolved.

The_implementation skips searching in the bases, when a lookup in the
bases has a cycle of dependency.  See the
section~\ref{sec:lookup-cycle}.

%%%%%%%%%%%%%%%%

\subsection{Other specific behabiors}

\subsubsection*{Class prefixes}

The_implementation accepts $\mathbf{encapsulated}$ only to a
long-class-specifier but errs/ignores to a short-class-specifier and a
der-class-specifier.  See Fact~\ref{encapsulated-class}.

The_implementation ignores $\mathbf{final}$ to a class.  See
Fact~\ref{final-class}.

\subsubsection*{Folding constants}

The_implementation uses int of SML for Integer values.  It may raises
exceptions on overflows in folding constants.

\subsubsection*{Constants and parameters}

The_implementation does not distinguish constants and parameters.
Both are treated as translation-time constants.  See
Fact~\ref{parameters}.

%%%%%%%%%%%%%%%%%%%%%%%%%%%%%%%%%%%%%%%%%%%%%%%%%%%%%%%%%%%%%%%%

\section{Instantiating}

%%%%%%%%%%%%%%%%

\subsection{Instantiating by one big pass}

{Instantiating} is a bit complex procedure.  It works top down in
order to handle modifiers and inner-outer matching, while it is
usually natural to work bottom up because each submodel only knows and
accesses their components but does not care its users.  In addition,
chains of dependency in name resolutions and simultaneous constant
folding complicate the procedure.  That is, determining array
dimensions needs folding constants which extensively accesses
constants in the components not yet instantiated.

Processing packages also works top down, although processing is
requested on demand at references.  It needs to work top down, because
it is necessary first to process the enclosing package to obtain a
class definition when the enclosing package has modifiers.  Here, the
enclosing relation is considered as extended by importing and
extending.

Note that lexically enclosing classes cannot affect the enclosed class
(See Fact~\ref{lexical-enclosing}).  Thus, it is safe to use the
textual definition for such class references.  For example,
Modelica.Fluid.Vessels.OpenTank is as textually defined, and it does
not need to process Modelica.Fluid.Vessels first to take the
definition.  Actually, the_implementation depends on this
shortcutting, that is necessary to process ".Modelica", or it would
make a cycle otherwise.

\subsection{Name resolutions in steps}

Name resolution needs be processed from the root for the hierarchies
of both packages and instances.  The_implementation splits name
resolution in two steps.  The first step resolves classes of
importing/extending.  The second step resolves classes of components.

(?) There is a simple dependency in processing the
$\mathit{.Modelica}$ package.  $\mathit{.Modelica.Icons.Package}$ is a
base of it.  Apparently, searching it needs to search $\mathit{Icons}$
and $\mathit{.Modelica}$.  See Section~\ref{lookup-shortcutting}.
Assume it would try continuing component processing.
$\mathit{Complex}$ is used in $\mathit{.Modelica.Icons}$.  It results
in searching $\mathit{Complex}$ in $\mathit{.Modelica}$, again.

%%%%%%%%%%%%%%%%

\subsection{Instances}

The_syntaxer does not distinguish classes and instances, and a class
is an instance when it is associated to a state variable.  It is like
classless languages as Self.

%%%%%%%%%%%%%%%%

\subsection{Handling arrays with modifiers}

The_implementation uses a pseudo function $\mathit{split}$ when a
non-each modifier is moved in a class definition that is an array.
For example, a modifier will be moved in the class definition in the
example below.  The indexing $v[i]$ is represented by a
$\mathit{split}$.  The index $i$ ranges in $\{1,\,...,\,10\}$.
%%%%
\begin{quote}
\begin{lstlisting}[aboveskip=-\baselineskip]
A a[10] (x=v)
==>
model A' = A (x=v[i])
A'[10] a;
\end{lstlisting}
\end{quote}

\begin{qquote} (\specref{7.2.5}\/) If the modified element is a vector
and the modifier does not contain the each-prefix, the modification is
split...\end{qquote}

$\mathit{split}$ is semantically an array indexing by a constant
index.  But, $\mathit{split}$ is not a proper syntax in Modelica,
because it only defines an array indexing on identifers but not on
expressions.

%%%%%%%%%%%%%%%%

\subsection{Handling inner/outer declarations}

The_syntaxer performs an inner-outer matching when it makes a list of
visible names in an instance.  It leaves an alias for an outer
variable pointing to an inner variable.  An alias is necessary to
retain the component relation for a connector (a place where a
connector is declared) which determines the flow direction.  Outer
aliases are replaced after handling connectors.  Replacing an outer
alias is performed by ignoring subscripts of a variable reference
because they are irrelevant.

%%%%%%%%%%%%%%%%

\subsection{Record class identity}

A record class needs identity for the purposes such as passing
arguments to functions.  The_implementation identifies a class by the
definition that is (last) modified by redeclarations.  Redeclarations
may change the components of a class, and the_implementation treats
redeclarations create a separate class but non-redeclarations do not
create a new one.  The_implementation creates an identity by
concatenating an identity of a package and a name given after
redeclarations.  See Fact~\ref{record-class-identity}.
The_implementation does not identify functions this way, because
modifications of values change the function.

%%%%%%%%%%%%%%%%%%%%%%%%%%%%%%%%%%%%%%%%%%%%%%%%%%%%%%%%%%%%%%%%

\section{Syntaxing}

{Syntaxing} performs simple semantic transformations to make a flat
model.  It rewrites connect_equations to equations.  In that
rewriting, it expands if-equations and for-equations that include
connect_equations.

%%%%%%%%%%%%%%%%

\subsection{Implicit Iterator Ranges}

A range of an implicit iterator is taken as a subscript part of an
expression that includes an iterator.  It checks the occurrence only
but ignores the surrounding expressions if any.  See
Section~\ref{sec:implicit-iterator-ranges}.

%%%%%%%%%%%%%%%%

\subsection{Expandable_connectors}

Instantiation of expandable_connectors is delayed, while other (usual)
classes are instantiated by descending the model hierarchy.  Expanding
expandable_connectors is performed in steps starting from a connector
at the shallowest position.

%%%%%%%%%%%%%%%%

\subsection{Connector operators}

Operator applications on or related to connectors,
$\mathrm{inStream}\,(v)$, $\mathrm{actualStream}\,(v)$, and
$\mathrm{cardinality}\,(c)$ are replaced by new variables with some
associated equations.  See a $\mathit{mix}$ variable in \specref{15}.
Variables are suffixed by \verb|_mix_in_| for
$\mathrm{inStream}\,(v)$, \verb|_actual_in_| for
$\mathrm{actualStream}\,(v)$, and \verb|_cardinality_| for
$\mathrm{cardinality}\,(c)$.

%%%%%%%%%%%%%%%%%%%%%%%%%%%%%%%%%%%%%%%%%%%%%%%%%%%%%%%%%%%%%%%%

\section{Clarification of the_specification}

Some rules are assumed in the_implementation on syntaxing.

%%%%%%%%%%%%%%%%

\subsection{Variable references in general}

\subsubsection*{Composite names}

A variable (instance) reference is in the form like
$x[...].y[...].z[...]$.  Subscripts are optional in a variable
reference.  A class reference is like $A.B.C$.  Subscripts are not
allowed in a class reference.  Classes $A$, $B$, and $C$ can be
packages or non-packages.  A constant refererence in a class is like
$A.B.C.a[...].b[...].c[..]$, where all $a$, $b$, and $c$ are constants
and subscripts are optional.

A (non-constant) variable reference does not have any prefixes (none
of a dot nor packages).  A class reference may be with a preceding
dot, which refers to the unnamed root.  Or, a class reference may
follow a variable reference, in that case classes are considered as
packages.

\begin{QRULE}\label{mixed-class-instance-reference} There is a single
namespace for classes and variables.  It is allowed to refer to a
constant in a package via a variable reference as $x.P.c$.  But, it is
illegal to refer to a class via a variable reference in a declaration
as (illegal) $x.C~v$.\end{QRULE}

\begin{QRULE} A global variable reference cannot have a preceding dot.
Note that $\mathit{time}$ is the only variable.\end{QRULE}
%%%%
\begin{quote}
\begin{lstlisting}[aboveskip=-\baselineskip]
/*ILLEGAL*/
model M
    Real x;
    equation
    x = .time;
end M;
\end{lstlisting}
\end{quote}

\begin{QFACT} An enclosing class is considered as a package in the
nesting of class definitions, and the scope of an enclosing class only
includes the constants but variables are ignored.\end{QFACT}
%%%%
See Rule~\ref{model-nesting} for the binary roles of a class.

\if 0
%%%% Rewrite then include to the body (AHO 2020-02-04)
\fi

\subsubsection*{Scopes}

The scope is flat in most cases, spanning from the main (non-base)
class to base classes where the class and variable names are distinct.
But, class names and constants are visible via enclosing classes.  An
enclosing class is defined to each (base) class.

%%%%%%%%%%%%%%%%

\subsection{Importing}

\def\baseofbase{This inclusion recurses transitively to the base
classes}

\def\noimport{This inclusion does not include the imported names in
the base classes nor the names in the enclosing classes in the base
classes}

Importing makes visible the names of the definition/declaration
elements of the specified package to the present class.  The inclusion
is from any classes but they are processed as packages.  The inclusion
is limited to the public elements. \baseofbase. \noimport.

\subsubsection*{Import-clause syntax}

\def\packagenames{P_0.\,...\,.P_{n-1}}

See \specref{13.2.1}.  A package name in an import-clause is
considered as fully~qualified.  It can import only from packages.  In
the following, $\mathit{package}$ is a sequence of identifiers
$\mathit{package}=\packagenames$.  Each prefix in $P_0\,...\,.P_i$
refers to a class, and $P_{n-1}$ refers to a package.
$\mathit{package}$ can be empty.  Each $P_i$ and $N$ refers to an
indentifier.

\begin{quote}
\begin{description}

\item [$\mathbf{import}\;N'=\mathit{package}.N$] (renaming
import/renaming single definition import): makes a class (including a
package) or a constant $N$ visible as an identifier $N'$.  There is no
syntactic difference when $N$ refers to a class/package or a constant.

\item [$\mathbf{import}\;\mathit{package}.*$] (unqualified import):
imports each public name visible in the package including one in the
base_classes as "$\mathbf{import}\;N_i=\mathit{package}.N_i$" for
$N_i$.  The package part can not be empty in this case.

\item [$\mathbf{import}\;\mathit{package}.N$] (qualified import/single
definition import): is equivalent to
"$\mathbf{import}\;N=\mathit{package}.N$".

\item [$\mathbf{import}\;\mathit{package}.\{N_0,N_1,...,N_{m-1}\}$]
(multiple definition import): is equivalent to
"$\mathbf{import}\;N_i=\mathit{package}.N_i$" for each $i$.

\if0
\item [$\mathbf{import}\;\mathit{package}$] (qualified import): is
equivalent to "$\mathbf{import}\;P_{n-1}=\packagenames$".
\fi

\if0
\item [$\mathbf{import}\;N=\mathit{package}$] (renaming import): makes
a package $P_{n-1}$ visible as an identifier $N$.
\fi

\end{description}
\end{quote}

%%%%%%%%%%%%%%%%

\subsection{Extending}

Extending includes the (definition/declaration) elements of the base
classes to the present class.  \baseofbase.  \noimport.  The
visibility is nothing to do with the bases because it is either
$\mathbf{public}$ or $\mathbf{protected}$.

\subsubsection*{Checks for extends clauses}

\begin{QRULE}\label{uniqueness-after-extension} It is an error if a
base class $A$ of $\mathbf{extends}\;A$ is defined in the class $A$
itself or in other base classes.  This is broadly interpreted as a
lookup of $A$ skips extends-clauses. \end{QRULE}

This may be a result of requiring the uniqueness of resolution of a
class name.  When another class $A$ were defined in $A$ in the code
below, two occurrences of $A$ would refer to the different classes,
which contradicts to an intuitive assumption that the both $A$ are the
same.
%%%%
\begin{quote}
\begin{lstlisting}[aboveskip=-\baselineskip]
model M
    extends A;
    A a;
end M;
\end{lstlisting}
\end{quote}

\begin{qquote} (\specref{5.6.1.4}\/) The classes of extends-clauses
are looked up before and after handling extends-clauses; and it is an
error if those lookups generate different results.\end{qquote}

\subsubsection*{Mutual dependence of extends-clauses}

\begin{QRULE}\label{ban-mutual-inheritance} Mutual dependence of
extends-clauses is prohibited.  \end{QRULE}

(?) The_specification does not likely state it explicitly.
The_specification states that inheritance is unified in cases where a
class extends a common base_class through distinct intermediate
base_classes.  Mutual dependence could be resolved by identifying.

%% ("diamond"-paths of inheritance)

\subsubsection*{Extending a single base multiple times (?)}

\begin{QRULE}\label{multiple-extensions-of-single-base} (?) Extending
a single class multiple times (usually via a common base class) is
allowed only if the modifiers to it are identical.  It is an error if
the modifiers differ.  \end{QRULE}

It is ambiguous what identical means.  Textual equality may not be
sufficient with class modifications.  Consider "extends~A(x=a)" and
"extends~A(x=b)" with "a=0" and "b=0".

(?) It may require {unifying inherited elements}.  Classes and
variables multiply inherited are unified.

\begin{qquote}\label{quo:unifying-inherited-elements} (\specref{5.6.1.4},
in the paragraph {\it{}The inherited contents of the element\/}) At
the end, the current instance is checked whether their children with
the same name are identical and only the first one of them is kept.
It is an error if they are not identical. (* A "child" means a
component. *)\end{qquote}

(* The_implementation does not check the equivalence of modifiers, and
arbitrary selects one modification. *)

\subsubsection*{Operator_record}

Operator_records have restrictions on extends-clauses.

\begin{qquote} (\specref{4.6}, in the table) (?) It is not legal to
extend from any of its enclosing scopes.  (* {\it{}enclosing\/} likely
refers to {\it{}enclosed} *).\end{qquote}

\subsubsection*{No scopes of bases in a lookup of a base class}

\begin{QRULE} A lookup of a base class name skips bases of the present
class.  It applies to a lookup of both the first part and the
remaining parts of a composite name. \end{QRULE}

%%%%
\begin{quote}
\begin{lstlisting}[aboveskip=-\baselineskip]
/*ILLEGAL*/

model A
    model B end B;
end A;

model M0
    extends B;
    extends A;
end M0;

model M1
    extends M1.B;
    extends A;
end M1;
\end{lstlisting}
\end{quote}
%%%%

\paragraph {Remark}

Note the first case is accepted with warnings in some_implementations,
and the both cases will be accepted, when the extends-clauses are
swapped, i.e., when $\mathbf{extends}\;A$ comes early.

\subsubsection*{Unifying defined classes}

\begin{QRULE}\label{multiple-definitions} It is not an error if
classes with the same name are defined multiple times in the main
class and its bases, when the definitions are considered identical.
Identical classes are a sort of unified.  It is defined in
the_specification that classes are identical if they are textually
identical after possible modifications. \end{QRULE}

For example, in Fluid/Interfaces, the declarations of
$\mathit{Medium}$ need be unified, when $\mathit{PartialPump}$ extends
both $\mathit{PartialTwoPort}$ and $\mathit{PartialLumpedVolume}$.
%%%%
\begin{quote}
\begin{lstlisting}[aboveskip=-\baselineskip]
partial model PartialTwoPort
    replaceable package Medium =
        Modelica.Media.Interfaces.PartialMedium;
    ......
end PartialTwoPort;

partial model PartialLumpedVolume
    replaceable package Medium =
        Modelica.Media.Interfaces.PartialMedium;
    ......
end PartialLumpedVolume;
\end{lstlisting}
\end{quote}
%%%%
$\mathit{PartialPump}$ is defined in
$\mathit{Modelica.Fluid.Machines.BaseClasses}$.\\
$\mathit{PartialTwoPort}$ and $\mathit{PartialLumpedVolume}$ are
defined in $\mathit{Modelica.Fluid.Interfaces}$.

\subsubsection*{Extending simple_types}

\begin{QFACT}\label{extending-simple-types} Extending simple_types is
legal either by a short_class_definition or by an extends-clause.  But
adding elements is illegal.\end{QFACT}

\subsubsection*{Restrictions on elements}

\begin{qquote} (\specref{4.5.2}\/) It is not legal to combine other
components or base-classes with an extends from an array class, a
class with non-empty base-prefix, a simple_type...\end{qquote}

%%%%%%%%%%%%%%%%

\subsection{A lookup of a class name}

Resovling class names is better understood not by a scope but by a
lookup procedure.  A lookup of a class name happens in import-clauses,
extends-clauses, component declarations, and composite name references
in expressions.  A lookup of import-clauses and extends-clauses has a
different procedure, and is performed in this order because one
depends on the predecessor.  A lookup is performed after applying
renaming modifiers/redeclarations to the present class.

A lookup of a composite class name starts from a lookup of the first
part followed by lookups of the remaining parts.  A lookup of the
remaining parts is performed in the class found in the previous
lookup.

A lookup considers the scopes of the following:
%%%%
\begin{itemize}

\item declared_elements~(s0).  A scope of the declared_elements of the
present class is closed in the class definition.

\item imported_names~(s1).  A scope of the imported_names transitively
extends to the declared_elements of the base_classes.

\item base_classes~(s2).  A scope of the base_classes transitively
extends to the declared_elements of the base_classes.

\item an enclosing class~(s3).  A scope of an enclosing class consists
of the declared_elements, the imported_names, and the base_classes,
and transitively extends to its enclosing classes.

\end{itemize}

\begin{QFACT}\label{lookup-shortcutting} A lookup first tries to find
in the declared_elements~(s0), and if it finds a name, other scopes
are skipped.  See \quoteref{uniqueness-of-name-lookup}.\end{QFACT}
%%%%
Especially, this rule is needed to search for $\mathit{Icons}$ in
$\mathit{.Modelica}$ which extends $\mathit{.Modelica.Icons.Package}$.
The scope of $\mathit{.Modelica}$ should include
$\mathit{.Modelica.Icons.Package}$ but it avoids a cycle by skipping
the base classes.

\begin{QFACT} A lookup of a first part of a name for an import-clause
and an extends-clause is not affected by redeclarations.  See
Rule~\ref{imported-not-replaceable} and
Rule~\ref{base-not-replaceable}.  A lookup respects class
redeclarations for an extends-clause in some
implementations.\end{QFACT}

\begin{QFACT}\label{lookup-renaming} A lookup of the remaining parts
respects class redeclarations in modifiers and in elements.  It also
respects outer prefixes.\end{QFACT}

%%%%%%%%%%%%%%%%

\subsection{A lookup for importing}

A package~name here denotes a class name of an import-clause.

\begin{QRULE}\label{lookup-imports} A lookup of the first part of a
package name is performed in the root of the namespace (the
unnamed-enclosing-class), since the package name is considered as
fully-qualified.  The root has no imports nor bases~(s0).

A lookup of the remaining parts is performed in the declared_elements
and in the base classes, but not in the imported names nor in the
enclosing classes~(s0+s2).  A lookup of a remaining part has a
restriction.  See Rule~\ref{lookup-cycle}.\end{QRULE}

%%%%%%%%%%%%%%%%

\subsection{A lookup for extending}

\begin{QRULE}\label{lookup-bases} A lookup of the first part of a name
of a base class does not include the base classes, but does include
the imported_names and the enclosing classes~(s0+s1+s3).  Note that
the base classes of the imported classes and the base classes of the
enclosing classes are included.  The names in the enclosing classes
can be hidded by the other names.

A lookup of the remaining parts is performed in the declared_elements
and in the base_classes, but not in the imported_names nor in the
enclosing classes~(s0+s2).  A lookup of a remaining part has a
restriction.  See Rule~\ref{lookup-cycle}.\end{QRULE}

\paragraph {Remark}

Some_implementations find a base_class through another extends-clause,
although it reports the code is illegal.
%%%%
\begin{qquote}\label{uniqueness-of-name-lookup}
(\specref{5.6.1.4}; in the paragraph {"The inherited contents of the
element"}) ... The classes of extends-clauses are looked up before and
after handling extends-clauses; and it is an error if those lookups
generate different results.\end{qquote}
%%%%
A lookup in the base_classes is only necessary for error detection.  A
lookup in the base_classes can simply be ignored in a lookup of a
class of an extends-clause.

\subsubsection*{Ignoring base_classes}

A lookup of a base_class name searches in the imported_names but may
not search in the base_classes in some cases.

%%%%
\begin{QRULE}\label{lookup-base-class} {(A lookup of a base_class)} A
lookup of a base_class name ignores the base_classes in the present
class and the element classes transitively, where the present class is
the class in which the lookup starts.  A lookup also likely ignores
mutually dependent extends-clauses. \end{QRULE}
%%%%

Ignoring the base classes in the present class is a consequence of the
uniqueness rule of extends-clauses (in
\quoteref{uniqueness-of-name-lookup}).  It may also be deduced from
the flattening process (\specref{5.6}\/).  The description in
{\it{}The local contents of the element\/} states the element classes
are inserted into the instance tree whose extends-clauses seems to be
empty at the time.  The lookup of a class of an extends-clauses of the
present class is perfomed with that state of the instance tree.

%%%%%%%%%%%%%%%%

\subsection{A lookup of a component class}

A lookup of a class name of a component declaration, or a variable
name or a class name in an equation or a statement starts at the
class, where a component/equation/statement is in.

\begin{QRULE}\label{lookup-components} A lookup of the first part of a
component class name is performed in the declared_elements, in the
imported_names, in the base_classes, and the enclosing
classes~(s0+s1+s2+s3).

A lookup of the remaining parts is performed in the declared_elements
and in the base_classes, but not in the imported_names nor in the
enclosing classes~(s0+s2).
\end{QRULE}

%%%%%%%%%%%%%%%%

\subsection{A short-cut in a lookup}\label{sec:lookup-cycle}

\subsubsection*{A cycle in a lookup for importing/extending}

It is common to import/extend a package defined in the class or in its
subpackages.  Such a package reference, when it appears in a remaining
part of a composite name, may make a circular dependency.  It is due
to the rule for a lookup of a remaining part, in
Rule~\ref{lookup-imports} or Rule~\ref{lookup-bases}.

Consider processing a toplevel package $M$ below.  A lookup of $M.I.P$
in $M$ first finds $M$, and then a lookup continues for $I$ in $M$.
It may need to look into bases of $M$, because $I$ is a remaining part
of the name.  However, it is problematic because resolving the bases
of $M$ is underway.
%%%%
\begin{quote}
\begin{lstlisting}[aboveskip=-\baselineskip]
package M
    package I
        package P end P;
    end I;
    extends M.I.P;
end M;
\end{lstlisting}
\end{quote}
%%%%
A similar code can be found in MSL~3.2.3, where $M$ as Modelica, $I$
as Icon, and $P$ as Package.

Similarly for importing, searching $C$ in $M.M.W$ may need to look
into the imported definitions.
%%%%
\begin{quote}
\begin{lstlisting}[aboveskip=-\baselineskip]
within M.M;
package W
    package C ... end C;
    import M.M.W.C.s;
end W;
\end{lstlisting}
\end{quote}
%%%%
A similar code can be found in MSL~3.2.3, where $M.M$ as
Modelica.Media, $W$ as Water, and $C$ as ConstantPropertyLiquidWater.

\subsubsection*{A short-cut in a lookup for an element class}

\begin{QRULE}\label{lexical-enclosing} Lexically enclosing
classes (the relation that is not extended by importing and extending)
cannot affect the enclosed class by modifications.  Thus, it is safe
to use the textual definition for such classes.\end{QRULE}
%%%%
For example, consider the case $M$ extends $M.I.P$ above.  As $M.I$ is
textually defined, it is not necessary to process $M$ first to take
the definition of $M.I$.  Thus, it avoids a potential cycle in the
definition.

%%(Actually, redeclarations may change the elements of a base class, but
%%they work on the textual definition of the base class).

%%%%%%%%%%%%%%%%

%% The above observation implies a rule:
%%%%
%% \begin{QRULE}\label{lookup-cycle} A lookup of a remaining part of a
%% composite name, that is for an importing/extending class, is
%% restricted to declared elements (s0), when the lookup happens to be in
%% the class in which it is originally started.\end{QRULE}

%% (*WRONG*)
%% \subsection{A lookup of a redeclared name}
%%
%% A {\classrenaming} can affect a lookup of a class name.
%%
%% \begin{QFACT}\label{lookup-redeclared-name} ({Lookup of a redeclared
%% name}) When a lookup finds a {\classrenaming} in the middle of a
%% composite name, a lookup continues after a lookup of the renaming by
%% suffixing with the remaining parts. \end{QFACT}
%%
%% For example, in searching "A.B.C.D", when a lookup finds a renaming
%% "C=X.Y" in A.B, the lookup restarts for "X.Y.D" in A.B\@.

%%%%%%%%%%%%%%%%

\subsection{Binary roles of a class}

A class is used both as a type of an instance or as a package.  It is
usually instantiated directly or via extends-clauses.  It is
considered as a package, when a class name appears in the middle of a
composite name.  Semantics of a class differs in these uses.

\subsubsection*{Model nesting}

A non-package class can contain a class, and it is possible to
instantiate an enclosed one.  Here, to instantiate a class means to
declare a variable of it or to use it as a target model.
%%%%
\begin{QRULE}\label{model-nesting} It is legal to instantiate a class
enclosed by a non-package class, but the accessible variables in the
enclosing class are restricted to constants.  This includes variables
used in modifiers.  That is, an enclosing class is processed as a
package in such a case, where declaring variables is permitted but
using them is prohibited. \end{QRULE}

It means that it is not possible to share a lexically visible variable
from enclosed classes as usually expected.

In the example below, assume instatiating $M$, where a class $M$
contains a class $M.A$.  Only constants in $M$ are accessible from
$A$, and it is illegal.  If $x$ were a constant, it is legal.  This
restriction also exists when using a class $M.A$ as a target model.
%%%%
\begin{quote}
\begin{lstlisting}[aboveskip=-\baselineskip]
/*ILLEGAL*/
model M
    /*constant*/ Real x=10;
    model A
        Real v=x;
    end A;
    A a;
end M;
\end{lstlisting}
\end{quote}

\subsubsection*{Processing a class as a package}

Rule~\ref{model-nesting} implies a fact:
%%%%
\begin{QFACT}\label{class-as-package} An enclosing class of a class is
processed as a package in a way that is different from a usual way.
Variable declarations and outer class definitions are treated as
non-existent when a class is processed as a package.  (?) Outer
constants are included.  (?) Is it legal to prefix constants by outer?
\end{QFACT}
%%%%
Note that a class is processed as a package during a lookup of
intermediate names in a composite name.

\begin{QRULE}\label{nonsense-element} Binary roles means allowance of
illegal elements.  For example, outer elements are illegal in a
package, but it is allowed unless it is used. \end{QRULE}

%%%%%%%%%%%%%%%%

\subsection{Modifications}

\subsubsection*{Modification syntax}

There is no syntactic difference of a modification on a class or a
variable: $A(...)$ or $x(...)$.  A class form applies to a declaration
$\mathbf{class}\;A\;...\mathbf{end}\;A$ or $\mathbf{class}\;A=...$,
and a variable form applies to a declaration $B\;x$.

Simple examples are:
%%%%
\begin{quote}
\begin{lstlisting}[aboveskip=-\baselineskip]
model A
    model B = C(..0..);
end A;

model M
    model D = A(B(..1..));
    /* D.B = C(..0.. + ..1..); */
end M;
\end{lstlisting}
\end{quote}

\begin{QFACT} The LHS of modifiers is an identifier (possibly, some
parenthesized modifiers may be attached to the LHS).  Especially, the
LHS does not admit array indexing.\end{QFACT}

\subsubsection*{Modifications to enumerations or {der}-classes}

\begin{QRULE}\label{modifications-to-der} (?) Modifications to
{der}-classes are illegal. \end{QRULE}

The simple_types accept modifiers to their attributes.  (?) For other
predefined_types such as $\mathit{Clock}$, $\mathit{StateSelect}$,
$\mathit{ExternalObject}$, $\mathit{AssertionLevel}$, and
$\mathit{Connections}$, it is not checked.  Note that the grammar in
the_specification defines initializers (e.g., in $C\;x=e$) as a form
of modification, and they modify the $\mathit{value}$ attribute.

\subsubsection*{Merging modifiers}

Muliple modifications can be applied to the same class.

\begin{QFACT}\label{each-final} (?) Modifications applied to the same
class have the same $\mathbf{each}$, $\mathbf{final}$,
$\mathbf{redeclare}$, and $\mathbf{replaceable}$ statuses. \end{QFACT}

\subsubsection*{Resolving names in modifiers}

Resolving names in modifiers differs in the LHS and the RHS of
modifiers.  The LHS of modifiers means $x$ in $x=e$ or in $x(e)$.
Note that the LHS is an identifier always.  Also, it differs between
modifiers in class~refinings and extends-clauses.

A rule of resolving names in the LHS of modifiers is:
%%%%
\begin{QRULE}\label{scope-in-modifiers} Resolving the LHS in modifiers
is done in the scope of the modified class ($a$ in $A$ in the example
below).  See {Example: Scope for modifiers} in \specref{7.2}.  The LHS
in the modifiers is visible as a derived class in an extends-clause,
while it is external in a class~refining.  Thus, protected elements
are not visible in a class~refining.  See (\specref{4.1}\/).
\end{QRULE}

A rule of resolving names in the RHS of modifiers is:
%%%%
\begin{QRULE}\label{scope-rhs-in-modifiers} Resolving the RHS is done
in the surrounding scope of the present class as usual.  It applies to
both classes and variables.  It implies name resolution is postponed
till expanding base classes. \end{QRULE}

For example, $a$ and $x$ in the modifier $A(a=x)$ are visible, where
$a$ is resolved in the LHS rule, and $x$ in the RHS rule.
%%%%
\begin{quote}
\begin{lstlisting}[aboveskip=-\baselineskip]
/*LEGAL*/
model A
    Real a=0;
    constant Real x=1;
end A;

model M0
    extends X;
    model X=A(a=x);
end M0;

model M1
    extends A(a=x);
end M1;
\end{lstlisting}
\end{quote}

It should be noted that modifiers to extends-redeclarations have
different scoping rules.  See
Rule~\ref{scope-rule-in-extends-redeclaration}.

\subsubsection*{Value_modifiers}

The components of the simple_types are specially call as attributes.

\begin{QFACT}\label{value-attribute} An attribute named
$\mathit{value}$ is defined as a parameter in the simple_types, which
holds a value of the type in the usual sense (the simple_types are
$\mathrm{Real}$, $\mathrm{Integer}$, $\mathrm{Boolean}$,
$\mathrm{String}$, and enumerations).  A modifier to the value is
usually used to define a new type of a simple_type with a default
value.  Specifying both a value_modifier and an initializer_modifier
simultaneously is illegal.\end{QFACT}
%%%%
\begin{quote}
\begin{lstlisting}[aboveskip=-\baselineskip]
/*ILLEGAL*/
model M
    Real v (value=1.0)=2.0;
end M;
\end{lstlisting}
\end{quote}

\subsubsection*{Independence of importing and extending from modifications}

\begin{QFACT}\label{renaming-in-resolving} Resolving a class of an
import-clause or extends-clause can be performed ignoring
modifications.  It is because a replaceable class is not allowed for
an import-clause or an extends-clause.\end{QFACT}

Note redeclarations (both class and variable) are type
parameterization.

\begin{qquote} (\specref{5.6.1.4}\/) The possible redeclaration of the
element itself takes effect (* in the step of "the element itself"
*).\end{qquote}

\begin{qquote} (\specref{5.6.1.4}\/) Extends clauses are not looked
up, but empty extends clause nodes are created and inserted into the
current instance~...\end{qquote}

\subsubsection*{Initializers}

Initializers attached to declarations like $C\;x=e$ are modifiers.  An
initializer_modifier to classes is a set of modifiers to each of the
components.  However, an initializer_modifier to the simple_types is
treaded as a modifier to the value attribute.  Despite the fact that
the simple_types are composite classes, their components (specially
called as attributes) are treated differently from other classes.  See
Quote~\ref{quo:attribute-access}.

\begin{QFACT} An initializer_modifier to a simple_type only modifies
the value attribute.\end{QFACT}

\begin{QRULE} An initializer_modifier needs to have the same
components in the RHS and LHS.  Especially, an initializer_modifier by
a base instance is not allowed when some components are
added.\end{QRULE}

An initializer_modifier $\mathit{x1}=\mathit{x0}$ below is illegal
because of an addition of $s$.  It is also illegal even if $s$ were a
constant.  Additions of non-component definitions are allowed.
%%%%
\begin{quote}
\begin{lstlisting}[aboveskip=-\baselineskip]
/*ILLEGAL*/

model B
    Real b=10;
end B;

model S
    extends B;
    Real s=20;
end S;

model M
    B x0;
    S x1=x0;
end M;
\end{lstlisting}
\end{quote}

\begin{QRULE} An initializer_modifier and modifiers can coexist.
Modifiers are ignored in that case.\end{QRULE}

A modifier $(x=20)$ is ignored in the following example.
%%%%
\begin{quote}
\begin{lstlisting}[aboveskip=-\baselineskip]
/*LEGAL*/
model A
    Real x=10;
    Real y=10;
end A;

model M
    A x0;
    A x1(x=20)=x0;      /* x1.x=10 */
end M;
\end{lstlisting}
\end{quote}

%%%%%%%%%%%%%

\subsection{Ordering of modifications}

\subsubsection*{Modifier ordering}

\begin{QRULE}\label{modifier-precedence} A latter application of
modifiers has a precedence.  The ordering is defined by a way how
implementations should look at the modifications.  A component
declaration is considered latter than a modifier
application.\end{QRULE}
%%%%
For example, $x0=10$ is effective over $X(\mathit{value}=20)$.
%%%%
\begin{quote}
\begin{lstlisting}[aboveskip=-\baselineskip]
model A
    model X=Real;
    X x0=10;		/* a.x0=10 */
    X x1;		/* a.x1=20 */
end A;

model M
    A a (X(value=20));
end M;
\end{lstlisting}
\end{quote}

\begin{qquote} (\specref{4.4.2}\/) The new environment is the result
of merging $\bullet$\,the modification of enclosing class
element-modification with the same name as the component
$\bullet$\,the modification of the component declaration in that
order.\end{qquote}

%% Modifcations are handled from the outside to inside of the
%% enclosing classes.  Modifcations to a class are handled from the
%% most derived class.  Also, modifications need to be applied each
%% time at handling each extends-clause.  That is, it is not allowed
%% expanding extends-clauses of base_classes in advance.  See
%% (\specref{5.6.1.4}\/), especially
%% step 4, it suggests the existence of the ordering.

\subsubsection*{Redeclaration ordering}

\begin{QFACT}\label{modification-ordering} Ordering of modifications
is inner to outer as strict applications.\end{QFACT}

In the following example, a redeclaration $X=D$ in elements is applied
later than and effective over $X=E$ in a modifier.
%%%%
\begin{quote}
\begin{lstlisting}[aboveskip=-\baselineskip]
model C Real v=10; end C;
model D Real v=20; end D;
model E Real v=30; end E;

model A
    replaceable model X=C;
    X a;        /* X=D */
end A;

model M
    redeclare replaceable model X=D;
    extends A (redeclare replaceable model X=E);
end M;
\end{lstlisting}
\end{quote}

In the following example, a redeclaration $X=E$ in $M$ is applied
later than and effective over $X=D$ in $B$.
%%%%
\begin{quote}
\begin{lstlisting}[aboveskip=-\baselineskip]
model C Real v=10; end C;
model D Real v=20; end D;
model E Real v=30; end E;

model A
    replaceable model X=C;
    X a;        /* X=E */
end A;

model B
    redeclare replaceable model X=D;
    extends A;
end B;

model M
    redeclare replaceable model X=E;
    extends B;
end M;
\end{lstlisting}
\end{quote}

\subsubsection*{Modifiers over redeclarations}

%% \begin{QRULE} Redeclaring a component does not drop the modifiers
%% attached to the original component declaration.\end{QRULE}

\begin{QRULE}\label{modifier-over-redeclaration} A modifier attached
to a replaceable is effective after a redeclaration on an instance.
In contrast, A modifier is dropped after a redeclaration on a
class.\end{QRULE}

For example, a modifier $x=30$ to a variable is effective after a
redeclaration $D~a$.

\begin{quote}
\begin{lstlisting}[aboveskip=-\baselineskip]
model C Real v=10; end C;
model D Real v=20; end D;

model A
    replaceable C a (v=30);
end A;

model M
    extends A (redeclare D a);
    /* a.v=30 */
end M;
\end{lstlisting}
\end{quote}

In contrast, a modifier $v=30$ to a class is dropped by a
redeclaration $X=D$.

\begin{quote}
\begin{lstlisting}[aboveskip=-\baselineskip]
model C Real v=10; end C;
model D Real v=20; end D;

model A
    replaceable model X=C (v=30);
    X a;
end A;

model M
    redeclare model X=D;
    extends A;
    /* a.v=20 */
end M;
\end{lstlisting}
\end{quote}

\paragraph {Remark}

Furthermore, in some_implementations, a modifier $v=30$ to a class is
kept when a redeclaration is in a modifier.

\begin{quote}
\begin{lstlisting}[aboveskip=-\baselineskip]
model C Real v=10; end C;
model D Real v=20; end D;

model A
    replaceable model X=C (v=30);
    X a;
end A;

model M
    extends A (redeclare model X=D);
    /* a.v=30 */
end M;
\end{lstlisting}
\end{quote}

%%%%%%%%%%%%%%%%

\subsection{Redeclarations}

A class redeclaration redefines a class, and a component redeclaration
redeclares a variable to be another class.  Or, a redeclaration
changes dimension sizes.  Redefinitions and redeclarations may be
jointly called redeclarations.

\begin{qquote} (\specref{7.2}\/) A more dramatic change is to modify
the type and/or the prefixes and possibly the dimension sizes of a
declared element. This kind of modification is called a
redeclaration...\end{qquote}

\begin{QFACT} $\mathbf{redeclare}$ is always required for a class
definition in modifiers. \end{QFACT}

Simply, it is an error to omit $\mathbf{redeclare}$ in modifiers.
%%%%
\begin{quote}
\begin{lstlisting}[aboveskip=-\baselineskip]
/*ILLEGAL*/

model C Real v=10; end C;
model D Real v=20; end D;

model A
    X a;
end A;

model M
    extends A (/*redeclare*/ class X=D);
end M;
\end{lstlisting}
\end{quote}

\subsubsection*{A target of a redeclaration}

Note that a target (replaceable) class of a redeclaration cannot be
defined at the same class, which results in duplicate definitions.

\begin{QRULE} Redeclaring a class/variable in class elements applies
to its base classes.  Redeclaring a class/variable in modifiers
applies to the main class and its base classes.  In both cases, a
replaceable class/variable in bases can be redeclared.  It means that
redeclaring a class definition is visible from base classes, although
a usual class definition is not visible. \end{QRULE}

\begin{QFACT}\label{no-colocate-redeclare-replaceable} Co-locating
redeclare and replaceable definitions is duplicate definitions and an
error.  Thus, a replaceable definition should be an inherited class
(which is defined in a base class). \end{QFACT}

Redeclaring $X=D$ in $M$ is visible from a base class $A$.
%%%%
\begin{quote}
\begin{lstlisting}[aboveskip=-\baselineskip]
model C Real v=10; end C;
model D Real v=20; end D;

model A
    replaceable model X=C;
    X a; /* X=D */
end A;

model M
    redeclare model X=D;
    extends A;
end M;
\end{lstlisting}
\end{quote}

\begin{QFACT} A redeclaration replaces a definition of a replaceable.
A replaceable defined outside is not visible. \end{QFACT}

\begin{quote}
\begin{lstlisting}[aboveskip=-\baselineskip]
/*ILLEGAL*/

model C Real v=10; end C;
model D Real v=20; end D;

replaceable model X=C;

model A
    X a;
end A;
model M
    extends A (redeclare model X=D);
end M;
\end{lstlisting}
\end{quote}

\subsubsection*{Imported names being not replaceable}

\begin{QRULE}\label{imported-not-replaceable} Imported names are not
replaceable. \end{QRULE}

An imported $X$ is not repalceable.
%%%%
\begin{quote}
\begin{lstlisting}[aboveskip=-\baselineskip]
/*ILLEGAL*/

model C Real v=10; end C;
model D Real v=20; end D;

model A
    replaceable model X=C;
end A;

model B
    import X=A.X;
    X a;
end B;

model M
    redeclare model X=D;
    extends B;
end M;
\end{lstlisting}
\end{quote}
%%%%
To make it replaceable in the code above, it needs to be rewritten as
$\mathbf{import}\;A;$ and
$\mathbf{replaceable}\;\mathbf{class}\;X=A.X;$ in $B$.

\subsubsection*{Base names being not replaceable}

\begin{QRULE}\label{base-not-replaceable} A base class in an
extends-clause cannot be replaceable, implying it cannot be
redeclared.  (?) Each part of a composite name be non-replaceable.\end{QRULE}

%% (???) It applies to both via definitions (in class elements) and via
%% modifiers.

\begin{qquote} (\specref{7.1.4}\/) The class name used after extends
for base-classes and for constraining classes must use a class
reference considered transitively non-replaceable.\end{qquote}
%%%%
\begin{qquote} (\specref{7.3}\/) A redeclare construct in a
modifier~...  A redeclare construct as an element~...  Both redeclare
constructs work in the same way.\end{qquote}
%%%%
\begin{qquote} (\specref{7.3.1}\/) ...~In contrast to normal extends
it is not subject to the restriction that B should be transitively
non-replaceable~...\end{qquote}

\begin{QFACT} There is a trick to extend a replaceable class.  One can
extend a replaceable class once redeclares it to be not
replaceable.\end{QFACT}
%%%%
It is used for $\mathit{BaseProperties}$ in
Media.Examples.TwoPhaseWater.  It can be found near line 2542,
Media/package.mo (in MSL~3.2.3).

\begin{quote}
\begin{lstlisting}[aboveskip=-\baselineskip]
  package TwoPhaseWater
    extends Modelica.Media.Water.StandardWater;
    redeclare model extends BaseProperties
      "Make StandardWater.BaseProperties non replaceable in order that
       inheritance is possible in model ExtendedProperties"
    end BaseProperties;

    model ExtendedProperties
      extends BaseProperties;
    ...
\end{lstlisting}
\end{quote}

\paragraph {Remark}

The both of the following code are illegal.  Some_implementations
accept the first code.  $X$ is redeclared, and $A$ extends $X=D$.
some_implementations accept a redeclaration in modifiers, but do not
accept a redeclaration in elements.  Some_implementations accept the
second code.  The redeclaration is ignored.  $X$ is not redeclared,
and $A$ extends $X=C$.

\begin{quote}
\begin{lstlisting}[aboveskip=-\baselineskip]
/*ILLEGAL*/

model C Real v=10; end C;
model D Real v=20; end D;

model A
    replaceable model X=C;
    extends X; /* X=D */
end A;

model M
    A a (redeclare model X=D);
end M;
\end{lstlisting}
\end{quote}

\begin{quote}
\begin{lstlisting}[aboveskip=-\baselineskip]
/*ILLEGAL*/

model C Real v=10; end C;
model D Real v=20; end D;

model A
    replaceable model X=C;
    extends X; /* X=C */
end A;

model M
    redeclare model X=D;
    extends A;
end M;
\end{lstlisting}
\end{quote}

\subsubsection*{Redeclarations of dimension sizes}

Redeclaring dimension sizes works non-$\mathbf{replaceable}$ elements.

\begin{qquote} (\specref{7.3.3}\/) Array dimensions may be redeclared;
provided the sub-typing rules in 6.3 are satisfied. [This is one
example of redeclare of non-replaceable elements.]\end{qquote}

\subsubsection*{Redeclarations to a class and a component}

\begin{QFACT} Redeclaring a class and redefining a component are
independent and both are effective.  \end{QFACT}

For example, $X\;a$ is converted to $D\;a$.
%%%%
\begin{quote}
\begin{lstlisting}[aboveskip=-\baselineskip]
model A Real v=10; end A;
model B Real v=20; end B;
model C Real v=30; end C;
model D Real v=40; end D;

model H
    replaceable model X=A;
    replaceable model Y=B;
    replaceable X a;
end H;

model M
    redeclare Y a;
    redeclare model X=C;
    redeclare model Y=D;
    extends H; /* a.v=40 */
end M;
\end{lstlisting}
\end{quote}

%%%%%%%%%%%%%%%%

\subsection{Redeclarations in modifiers and in elements}

%%Redeclarations can be written in modifiers or in class elements.

Redeclarations can be written in class elements (in-element) or in
modifiers (in-modifier).  They are sightly different
syntactically/semantically.  That is, an in-element redeclaration
accepts $\mathbf{inner}$/$\mathbf{outer}$, but an in-modifier
redeclaration not (see Fact~\ref{inner-outer-in-redeclarations}).  An
in-element redeclaration affects a base class, which do not happen in
usual languages.

\begin{QFACT} Coexisting an in-element redeclaration and an
in-modifier redeclaration in an extends-clause which would apply to
the same class/variable is illegal.\end{QFACT}
%%%%
This is a rephrasing of the following quote.
%%%%
\begin{qquote} (\specref{7.3}\/) The redeclare construct as an
element ... cannot be combined with a modifier of the same element in
the extends-clause.\end{qquote}

This does not apply to a combination of a class and a component, nor a
non-redeclaration modifier.  The modifier below has an effect as
$a.v=30$.
%%%%
\begin{quote}
\begin{lstlisting}[aboveskip=-\baselineskip]
model C Real v=10; end C;
model D Real v=20; end D;

model A
    replaceable model X=C;
    X a; /* a.v=30 */
end A;

model M
    redeclare replaceable model X=D;
    extends A (a.v=30);
end M;
\end{lstlisting}
\end{quote}

\subsubsection*{Ordering of redeclarations}

\begin{QRULE} A redeclaration in a derived class is applied later to a
redeclaration in a modifier to a base class (in-element vs.\@
in-modifier).  A redeclaration in a derived class is applied later to
a redeclaration in a base class (in-element vs.\@
in-element).\end{QRULE}

Redeclaring $X=D$ in $M$ has precedence to a modifier $X=E$.

\begin{quote}
\begin{lstlisting}[aboveskip=-\baselineskip]
model C Real v=10; end C;
model D Real v=20; end D;
model E Real v=30; end E;

model A
    replaceable model X=C;
    X a; /* X=D */
end A;

model M
    redeclare replaceable model X=D;
    extends A (redeclare replaceable model X=E);
end M;
\end{lstlisting}
\end{quote}

Redeclaring $X=E$ in $M$ has precedence to $X=D$ in $B$.
%%%%
\begin{quote}
\begin{lstlisting}[aboveskip=-\baselineskip]
model C Real v=10; end C;
model D Real v=20; end D;
model E Real v=30; end E;

model A
    replaceable model X=C;
    X a; /* X=E */
end A;

model B
    redeclare replaceable model X=D;
    extends A;
end B;

model M
    redeclare replaceable model X=E;
    extends B;
end M;
\end{lstlisting}
\end{quote}

\subsubsection*{Redeclarations with prefixes}

\begin{QFACT}\label{inner-outer-in-redeclarations} Replaceables accept
prefixes inner/outer.  In-element redeclarations accept inner/outer,
but in-modifier redeclarations do not.\end{QFACT}

%%%%%%%%%%%%%%%%

\subsection{Merging parts of a redeclaration}

A redeclaration merges a part of a replaceable (original) declaration.
It is called inheriting in the_specification.

\subsubsection*{Merging prefixes in a redeclaration}

\begin{qquote} (\specref{7.3}\/) In redeclarations some parts of the
original declaration is automatically inherited by the new
declaration.\end{qquote}

\begin{qquote} (\specref{6.3}\/) (i.e., no need to repeat 'parameter'
in a redeclaration).\end{qquote}

\begin{QFACT} A variability prefix is merged.  A prefix input or
output is merged.  Prefixes inner and/or outer are merged.  A
conditional part is merged.  (See \specref{6.3}\/).\end{QFACT}

The definition of merging is made precise by the class_compatibility
(See Quote~\ref{quo:class-compatibility}).

\begin{QRULE} (?) Visibility statuses must be equal.  Final statuses
must be equal (final variables can be replaceable).  A variability
prefix is replaced with regard to the variability ordering.  An input
or output prefix in a replaceable is added, when none are given in a
redeclaration.  Inner and/or outer prefixes in a replaceable are
added, when none are given in a redeclaration.  A conditional part is
added.  If both a replaceable and a redeclaration have a conditional
part, they must be semantically the same.\end{QRULE}

That is, $\mathit{parameter}$ on a replaceable replaces one if a
redeclaration is $\mathit{discrete}$ or $\mathit{continuous}$, but
does not replace $\mathit{constant}$.

\subsubsection*{Merging modifiers in a redeclaration}

The_implementation follows the rules of merging modifiers, when
modifiers are in a redeclaration.  Note that it may not be compatible
with some_implementations.

\begin{QRULE}\label{modifiers-in-redeclaration} Modifiers in a
redeclaration is merged to ones in a replaceable with modifiers in a
redeclaration being applied later.\end{QRULE}

\subsubsection*{More on merging parts of a redeclaration}

\paragraph {Remark}

Some_implementations do not merge inner/outer prefixes by a
redeclaration in elements.  That is, an outer prefix attached to $X$
in $A$ is dropped in a redeclaration in $B$, in the following example.
%%%%
\begin{quote}
\begin{lstlisting}[aboveskip=-\baselineskip]
model C Real v=10; end C;
model D Real v=20; end D;
model E Real v=30; end E;

model A
    outer replaceable model X=C;
    X a;
end A;

model B
    redeclare model X=D;
    extends A;
end B;

model M
    inner model X=E;
    B b;        /* b.a.v=20 */
end M;
\end{lstlisting}
\end{quote}

\paragraph {Remark}

In some_implementations, a redeclaration in modifiers and one in
elements work differently.  One in elements drops a modifier attached
on a replaceable, but one in modifiers does not.

A modifier $v=30$ to a class $C$ is dropped by a redeclaration in
elements, in the following code.
%%%%
\begin{quote}
\begin{lstlisting}[aboveskip=-\baselineskip]
model C Real v=10; end C;
model D Real v=20; end D;

model A
    replaceable model X=C (v=30);
    X x;
end A;

model A0=A (redeclare model X=D);

model A1
    redeclare model X=D;
    extends A;
end A1;

model A2
    extends A (redeclare model X=D);
end A2;

model M
    A0 a0;      /* a0.x.v=30 */
    A1 a1;      /* a1.x.v=20 */
    A2 a2;      /* a1.x.v=30 */
end M;
\end{lstlisting}
\end{quote}

A modifier $v=30$ to a component $x$ is dropped by a redeclaration in
elements, in the following code.
%%%%
\begin{quote}
\begin{lstlisting}[aboveskip=-\baselineskip]
model C Real v=10; end C;
model D Real v=20; end D;

model A
    replaceable C x (v=30);
end A;

model A0=A (redeclare D x);

model A1
    redeclare D x;
    extends A;
end A1;

model A2
    extends A (redeclare D x);
end A2;

model M
    A0 a0;      /* a0.x.v=30 */
    A1 a1;      /* a1.x.v=20 */
    A2 a2;      /* a1.x.v=30 */
end M;
\end{lstlisting}
\end{quote}

%%%%%%%%%%%%%%%%

\subsection{Extends-redeclarations}

An extends-redeclaration defines a new class based on an inherited
class and replaces it.  It redefines an inherited class as other
redeclarations.

\subsubsection*{A base being replaceable}

\begin{QFACT} An extends-redeclaration applies to an inherited class.
It implies that it searches for a base class in the base classes of
the present class where it is defined.  While a base class usually
should not be replaceable, a base class of an extends-redeclaration
needs to be replaceable.\end{QFACT}

\begin{qquote} (\specref{7.3.1}\/) A class declaration of the type
"$\mathbf{redeclare}\;\mathbf{class}\;\mathbf{extends}\;B(...)$", ...,
replaces the inherited class $B$ with another declaration that extends
the inherited class~...\end{qquote}

It is not straightforwardly translated to a combination of an
extends-clause and a redeclaration.  It is because the scoping rule of
modifiers is unusual.  In addition, an extends-redeclaration extends
an original base and replaces it, but a redeclaration may hide an
original base.

It is illegal to define one without $\mathbf{redeclare}$, such as
$\mathbf{class}\;\mathbf{extends}\;A~...~\mathbf{end}\;A$.  It is
simply a duplicate definition, when it were allowed.

%% Rule~\ref{no-colocate-redeclare-replaceable} implies that.

\subsubsection*{A scope of modifiers}

\begin{QRULE}\label{scope-rule-in-extends-redeclaration} The scope of
the RHS of a modifier to an extends-redeclaration is the class that is
to be redeclaring (the base and the body of the class).  But, it does
not include the class in which a modifier appears (the ususal scope).
Note that, the scope of the RHS of a modifier is not the same as the
LHS, because the LHS is limited to the base.\end{QRULE}

The modifier $(a=x+y)$ is legal in the following example, because $x$
and $y$ are visible.  In contrast, $z$ declared in $M$ is not visible,
if it were used in the modifier.

\begin{quote}
\begin{lstlisting}[aboveskip=-\baselineskip]
model E
    replaceable model A
	Real a=10;
	Real x=20;
    end A;
end E;

model M
    extends E;
    redeclare model extends A (a=x+y)
	Real y=30;
    end A;
    Real z=40;
    A m;
end M;
\end{lstlisting}
\end{quote}

This occurs in MSL.  A parameter $\mathit{preferredMediumStates}$ is
visible in modifiers, near line 154, Media/Water/package.mo (in
MSL~3.2.3).  But, $\mathit{preferredMediumStates}$ is defined in
$\mathit{BaseProperties}$ but not in $\mathit{WaterIF97\_base}$ nor
its base $\mathit{PartialTwoPhaseMedium}$.

\begin{quote}
\begin{lstlisting}[aboveskip=-\baselineskip]
partial package WaterIF97_base
  extends Interfaces.PartialTwoPhaseMedium(...);
  redeclare replaceable model extends BaseProperties(
    h(stateSelect=if ph_explicit and preferredMediumStates
           then StateSelect.prefer
           else StateSelect.default), ...)
\end{lstlisting}
\end{quote}

The definition of $\mathit{preferredMediumStates}$ can be found at
line 241, Media/package.mo (in MSL~3.2.3).

\begin{quote}
\begin{lstlisting}[aboveskip=-\baselineskip]
package Interfaces
  partial package PartialMedium
    replaceable partial model BaseProperties
      parameter Boolean preferredMediumStates=false
\end{lstlisting}
\end{quote}

%%%%%%%%%%%%%%%%

\subsection{Inner/outer declarations}

Inner-outer matching makes a class definition or a variable
declaration depend on instantiations.  It makes a name defined in an
outer instance visible in an inner instance.

%% A class (even without
%% modifiers) can have different classes for elements at each
%% instantiation.

\subsubsection*{An inner-outer relation}

Inner-outer refers to a relation of component embedding.  A word
{\it{}instance_hierarchy\/} is also used to refer to it.  Assume class
$M$ includes declarations $A\;a$ and $B\;b$, and $B$ includes a
declaration $A\;c$.  By instantiating $M\;m$, $m$ is outer and $m.b$
is inner.  When the variable $A\;a$ is prefixed by $\mathbf{inner}$
and $A\;c$ by $\mathbf{outer}$, the references $m.a$ and $m.b.c$
are identical.

Note that an enclosing class relation is not an embedding relation.
They are different concepts.

\begin{QRULE}\label{inner-outer} Inner/outer prefixes applies to both
classes and variables.  An outer definition/declaration imports an
identifier defined in some outer instance and makes it visible in the
present class.  An inner prefix indicates a definition/declaration to
match to an outer one.  An outer class/variable can be declared
multiple times in a class and its bases.  They refer to the same inner
class/variable. \end{QRULE}
%%%%
An outer prefixes an identifier, and an inner prefixes a class
definition or a variable declaration.

\begin{QFACT} (?) An inner-outer relation on classes does not extend
outward to lexically enclosing classes. \end{QFACT}

In the following example, the model class is $M$.  The outer $P.X$
is defined in a package $P$ and it cannot have an inner-outer
relation.  Note the hierarchies are $M\owns{}P.B$ and $P>P.B$ ($\owns$
for component embedding, $>$ for lexical enclosing).
%%%%
\begin{quote}
\begin{lstlisting}[aboveskip=-\baselineskip]
/*ILLEGAL*/

model A0 Real x=10; end A0;
model A1 Real x=20; end A1;

model P
    outer model X = A0;
    model B
       X a;
    end B;
end P;

model M
    inner model X = A1;
    P.B b;
    Real y = b.a.x;
end M;
\end{lstlisting}
\end{quote}

\subsubsection*{Class_compatibility}

There is a compatibility restriction among classes of inner-outer.
%%%%
\begin{QRULE}\label{inner-outer-compatibility} A relation
$A\,{\mathcode`-=`-\mathit{inner-outer}}\,B$ is allowed for classes,
if $A$ is a subclass of $B$, $A<:B$.  (?) What for other classes?  (?)
What for variable declarations?  \end{QRULE}

\begin{quote}
\begin{lstlisting}[aboveskip=-\baselineskip]
inner model C = A;
outer model C = B;
model A extends B; end A;
\end{lstlisting}
\end{quote}

An example of inner/outer prefixes on classes is given in
the_specification as function definitions (functions are a kind of
classes).  See \specref{5.4.1}.

\subsubsection*{Syntax restrictions of modifiers}

\begin{QFACT} Outer class definitions are short_class_definition.
Modifiers to outer definitions/declarations are usually ignored unless
there is no matching inner.  Inner definitions/declarations have no
restrictions.  Simultaneous inner/outer class definitions are
short_class_definitions.  Modifiers to simultaneous inner/outer
definitions/declarations apply to the inner part unless there is no
matching inner. \end{QFACT}
%%%%
See \specref{5.5}.

\subsubsection*{Outers in a class during lookups}

\begin{QFACT} Outer classes and constants are treated as non-existent
when a class is processed as a package. \end{QFACT}
%%%%
See also Fact~\ref{class-as-package}.

\subsubsection*{Extending outers}

\begin{QFACT} It is legal to extend an outer class. \end{QFACT}

\subsubsection*{Simultaneous inner/outer}

\begin{QFACT} A simultaneous inner/outer definition/declaration can be
regarded as being with inner and outer separately.  The only definite
effect is that the present class sees an outer
class/variable.\end{QFACT}
%%%%
See \specref{5.5}.

Note that the instance created for a simultaneous inner/outer
definition is not visible in the class it is defined (exactly
speaking, in case when there is some inner which an outer part
matches).

%%%%%%%%%%%%%

\subsection{Visibility}

\begin{QRULE} A protected class is visible from an enclosed
class.\end{QRULE}

In syntax, $\mathbf{public}$, $\mathbf{protect}$, $\mathbf{equation}$,
and $\mathbf{algorithm}$ construct a section separated by themselves
until the end of class elements.  It is public, when sections may be
started with none of them.  That is, sections of equations and
algorithms have no distinction of public/protected.

(?) (*CHECK*) Check the visibility of import/extends-clauses and
redeclarations.

%%%%%%%%%%%%%

\subsection{Enclosing classes}

A lookup of a name in a class sometimes continues to the enclosing
class of the class.

\subsubsection*{An enclosing class of a definition-body}

\begin{QFACT} An enclosing class is a class which defines the
definition-body of the requested class (assigning a name does not
change the enclosing class).  An enclosing class may be a modified
one, and thus, it is not just the lexical relation.  An enclosing
class respects the relation of a definition-body, that is, each
main/base class has a separate enclosing class.\end{QFACT}

\subsubsection*{Scopes not extending to enclosing classes}

Elements from enclosing classes are not visible through import- or
extends-clauses.  The following code is illegal, because $v$ is
visible from $A$, but not visible in the derived class $M0$.  Also,
$v$ is not visible in $M1$.
%%%%
\begin{quote}
\begin{lstlisting}[aboveskip=-\baselineskip]
/*ILLEGAL*/

package P
    constant Real v=10;
    model A end A;
end P;

model M0
    extends P.A;
    Real z=v;
end M0;

model M1
    import P.A;
    Real z=v;
end M1;
\end{lstlisting}
\end{quote}

\subsubsection*{A scope to each base class}

An enclosing class is associated to each base class.  Different
classes are visible by the same name in a class.

\begin{quote}
\begin{lstlisting}[aboveskip=-\baselineskip]
package B0
    type T = Real(value=10);
    model B1
        T b;
    end B1;
end B0;

package D0
    type T = Real(value=20);
    model D1
        extends B0.B1;
        T d;
    end D1;
end D0;

model M
    D0.D1 m; /* m.b=10, m.d=20 */
end M;
\end{lstlisting}
\end{quote}

\subsubsection*{An enclosing class is the defining class}

In the following, the enclosing class of $D0.B1$ is $D0|B0$, a base of
$D0$ (as denoting a base part of a class by $D0|B0$).

\begin{quote}
\begin{lstlisting}[aboveskip=-\baselineskip]
package B0
    type T = Real(value=10);
    model B1
        T b;
    end B1;
end B0;

package D0
    extends B0(T(value=20));
end D0;

model M
    D0.B1 m; /* m.b=20 */
end M;
\end{lstlisting}
\end{quote}

\subsubsection*{An enclosing class of a reclaration}

A redeclaration works as expected, as it does not change the scope.

\begin{quote}
\begin{lstlisting}[aboveskip=-\baselineskip]
package B0
    type T = Real(value=10);
    model B1
        replaceable model B2
            T x;
        end B2;
        B2 b;
    end B1;
end B0;

package D0
    type T = Real(value=20);
    model D1
        extends B0.B1;
        redeclare model B2=D2;
        model D2
            T x;
        end D2;
    end D1;
end D0;

model M
    D0.D1 m;  /* m.b.x=20 */
end M;
\end{lstlisting}
\end{quote}

%%%%%%%%%%%%%%%%

\subsection{Restrictions of within}

\subsubsection*{Processing a target model}

A target model is usually loaded from a user specified file.  It is
possible to modify a target model, when it is defined in an enclosing
classes which has modifiers on it. (?) It probably is not an
intention.
%%%%
\begin{QRULE} (?) It is forbidden to declare a class in an enclosing
class which is defined by refining, where an enclosing class is
specified via $\mathbf{within}$. \end{QRULE}
%%%%
That is, $A$ is not defined by $\mathbf{package}\;A = ...$ when using
$\mathbf{within}\;A$.

\begin{quote}
\begin{lstlisting}[aboveskip=-\baselineskip]
(*CHECK*)
/* File A: */
package A = B(M(x=10));

/* File M: */
within A;
model M
    Real x=0;
end M;
\end{lstlisting}
\end{quote}

%%%%%%%%%%%%%%%%

\subsection{Class prefixes}

\begin{QFACT}\label{encapsulated-class} (?) $\mathbf{encapsulated}$
can be attached to any class definitions in the syntax rules, but it
only has a meaning to long-class-specifier and meaningless to
short-class-specifier and der-class-specifier.\end{QFACT}

\begin{QFACT} $\mathbf{encapsulated}$ is only effective for main
(non-base) classes.\end{QFACT}

\begin{QFACT}\label{final-class} $\mathbf{final}$ to a class is
unclear.  It might disable modifiers or might disable extending.
\end{QFACT}

%%%%%%%%%%%%%%%%

\subsection{Variability}

Variability of components is specified by keywords for
$\mathbf{constant}$, $\mathbf{parameter}$, and $\mathbf{discrete}$,
but $\mathit{continuous}$ when nothing is specified.  Variability is
ordered by the inclusion relation: $$\mathit{constant} \subset
\mathit{parameter} \subset \mathit{discrete} \subset
\mathit{continuous}\;.$$ Continuous means unconstrained, and declaring
continuous integers is valid.  The ordering is used in class
compatibility.  It is described in \specref{3.8}\/ and \specref{6.3}.

\begin{QFACT}\label{parameters} The ordering strict.  However, it can
be treated as $\mathit{parameter}=\mathit{constant}$ for translation
and simulation, although parameters and constants are distinguished at
a surface level.  Note that parameters, in addition to constants, in
packages are visible.  Values of parameters and constants may be
specified by modifiers, unless they are final.\end{QFACT}
%%%%
Constants cannot be changed at runtime after translation of a model by
definition.  However, parameters are also hard to be changed at
runtime, because parameters are treated as constants in simplification
of constants.

\begin{qquote} (\specref{3.8.1}\/) The value of a constant can be
modified after it has been given a value, unless the constant is
declared final...\end{qquote}

\begin{qquote} (\specref{4.4.4}\/) A constant variable is similar to a
parameter with the difference that constants cannot be changed after
translation and usually not changed after they have been given a
value.\end{qquote}

\begin{qquote} (\specref{7.2.6}\/) [Setting the value of a parameter
in an experiment environment is conceptually treated as a
modification.\end{qquote}

\begin{QFACT} Variability is not orthogonal to number types.  Simple
types except Real are discrete.  However, declaring such as continuous
integers is allowed (and typical because continuous is the
default).\end{QFACT}
%%%%
See \specref{3.8.3}\/ for discrete values.

%%%%%%%%%%%%%%%%

\subsection{Connectors}

A connector is a type or a record that is marked to be used in
connect_equations.

\subsubsection*{Sides of connectors}

A side is an aspect of a connector whether it is connected in a
declaring class.  An outside_connector is an internal connection in
the declaring class, and the flow direction is considered reversed
(the value is negated) for that connection.  An opposite is an
inside_connector.  A connector is considered split to the
inside/outside ones, but they share the same variables.  The side is
taken into account in connect_equations but not in the {cardinality}
operator.

\begin{QFACT} Determination of a side of a connector must respect the
place where the connector is declared, and inner/outer declarations
are ignored for that.  An outside_connector is distinguished by the
use of a connector that is declared and used in the class.  That is,
the first part of a composite name is a declared connector.\end{QFACT}
%%%%
For example, $c$ in $\mathrm{connect}\,(c,...)$ is an
outside_connector when it appears in the class declaring $c$.  And
also, $c.m.d$ in $\mathrm{connect}\,(c.m.d,...)$ is an
outside_connector when it appears in the class declaring $c$, where
$m$ is a possibly empty list of components (connectors or
non-connectors).  (*CHECK*) It is an outside_connector in the case
when a non-connector intervenes like $c.m.d$.

\subsubsection*{A connection_set}

A connector is a record that may contain simple_types, records,
operator_records, and connectors (but not models specifically).  A
connect_equation connects variables including component connectors
declared in the connector.

A connection_set consists of tuples of a variable and its side.  The
set is defined as a symmetric and transitive closure of a relation
that contains the pairs in connect_equations.  The variables are
simple_types or operator_records (overdetermined_records), and a
record in a connector is broken down to its components, but an
overdetermined_record (or every operator_record (?)) is treated as a
single element.

\subsubsection*{(?) Connect equations on arrays}
(?) Array sizes are determined?

\subsubsection*{Connection restrictions}

Connect equations are fixed at translation-time.

\begin{qquote} (\specref{9.3}\/) $\bullet$\,The connect_equations (and
the special functions for overdetermined connectors) may only be used
in equations and may not be used inside if-equations with
non-parametric condition, or in when-equations. [For-equations always
have parameter expressions for the array expression.]\end{qquote}

\begin{qquote} (\specref{8.3.2}\/) The expression of a for-equation
shall be a vector expression. ... The expression of a for-equation
shall be a parameter expression.\end{qquote}

\subsubsection*{Expandable_connectors}

An expandable_connector is a set of connectors, and it is only
connected to expandable_connectors.  A component will be included when
it is used.  All components in an expandable_connector are implicitly
considered as connectors.  An expandable_connector is first considered
empty until a component (even a declared one) is included by using it
explicitly.

A direct reference of a component in a connect equation signifies an
addition of a component to an expandable_connector.  A reference is
$m...m.c.d$, with $m$ a possibly empty component, $c$ an
expandable_connector, and $d$ a component.

%% A component will be included for the outside use of an
%% expandable_connector even if it is not declared in the definition.

\begin{qquote} (\specref{9.1.3}\/) $\bullet$\,expandable connectors
can only be connected to other expandable connectors.\end{qquote}

\begin{qquote} (\specref{9.1.3}\/) $\bullet$\,All components in an
expandable connector are seen as connector instances even if they are
not declared as such\end{qquote}

\begin{qquote} (\specref{9.1.3}\/) Before generating connection
equations non-parameter scalar variables and non-parameter array
elements declared in expandable connectors are marked as only being
potentially present.\end{qquote}

(*CHECK*) Declared components of an expandable_connector cannot be
used in connect_equations unless they are connected to connectors not
in some expandable_connectors.

\subsubsection*{Stream_connectors}

See \specref{15}.

A stream_connector is a connector containing stream_variables.  A
stream_connector contains a single scalar real flow_variable and some
reals of (or arrays of) stream_variables.  The record components of a
stream_variable are considered as stream_variables.

$\mathrm{inStream}\,(v)$ and $\mathrm{actualStream}\,(v)$ take a
stream_variable (which may be an array).

\subsubsection*{Overdetermined_types/records}

A type/record defining $\mathrm{equalityConstraint}\,(x,y)$ is called
overdetermined_type/record.  The {Connections} class defines some
predefined functions to reduce overdetermined constraints.

\subsubsection*{(?) Operator_record components}

(?) A connection component of an operator_record is used as an
aggregate in generated equations with defined eqaulity, summation, or
negation.

\subsubsection*{Unconnected connectors}

Flow variables are set to zero for unconnected connectors.
Conditional connectors are unconnected if their conditions are false.
See Section~\ref{conditional-component}.

\subsubsection*{cardinality operator}

The $\mathrm{cardinality}\,(c)$ operator returns the count of the uses
of connect_equations on the inside part of the argument connector.  It
is not the size of a connection set.  Cardinality of an embedded
connector is counted, when it is indirectly used in connect_equations.
For example, $\mathrm{cardinality}\,(c.d)$ counts an equation
$\mathrm{connect}\,(c,\_)$.

{cardinality} only appears in equation sections (only in assertions).
{cardinality} does not take expandable_connectors, components of
expandable_connectors, nor arrays of connectors.  {cardinality} is a
translation-time constant.  However, $\mathrm{cardinality}\,(c)$
cannot be replaced by a constant in general, because a connector
reference is not a translation-time constant (array indices can be
non-constants).

The_clarification a bit differs from the_specification:
%%%%
\begin{specquote}{3.7.2} Returns the number of (inside and outside)
occurrences of connector instance c in a connect-equation as an
Integer number.  See also Section 3.7.2.3.\end{specquote}
%%%%
\begin{specquote}{3.7.2.3} The cardinality is counted after removing
conditional_components.  and may not be applied to expandable
connectors, elements in expandable connectors, or to arrays of
connectors (but can be applied to the scalar elements of array of
connectors).  The cardinality operator should only be used in the
condition of assert and if-statements -- that do not contain connect
(and similar operators -- see section 8.3.3).\end{specquote}

%%%%%%%%%%%%%%%%

\subsection{Conditional_components}\label{conditional-component}

A declaration may have a condition to enable/disable it.  A disabled
non-connector component is unusable.  A conditional connector can be
used only in connect_equations, and connect_equations with a disabled
connector are considered removed.  Connectors in a disabled component
are disabled.

\subsubsection*{Conditional non-connectors}

A disabled non-connector component is unusable.  Thus, usually,
conditional non-connector components are used for the purpose to
observe particular states in a simulation.

\subsubsection*{Restrictions of conditional connectors}

Conditional_components are only used in connections.
%%%%
\begin{specquote}{4.4.5} A component declared with a
condition-attribute can only be modified and/or used in
connections\end{specquote}
%%%%
\begin{specquote}{5.6.2} [Conditional_components can be used in
connect-statements, and if the component is conditionally disabled the
connect-statement is removed.]\end{specquote}
%%%%
Conditions of conditionals are translation-time constants.
%%%%
\begin{qquote} (\specref{4.4.5}\/) The expression must be a Boolean
scalar expression, and must be a parameter-expression\end{qquote}
%%%%
Conditions of conditionals are retained after component
redeclarations.
%%%%
\begin{qquote} (\specref{4.4.5}\/) A redeclaration of a component may
not include a condition attribute; and the condition attribute is kept
from the original declaration\end{qquote}
%%%%
Conditional_components cannot be used in modifiers of declarations.
It is indicated by the example in \specref{4.4.5}.
%%%%
Conditional_components cannot have ":" as an array dimension.
It is described in \specref{6.3}.

\subsubsection*{Conditional declarations}

\begin{QFACT}\label{scope-with-conditional-declaration} A condition of
a conditional_component does not affect the scope.  Particularly, a
variable with a condition can be a target of modifications regardless
of the condition.\end{QFACT}

Is the value of $x$ dependes on a condition?
%%%%
\begin{quote}
\begin{lstlisting}[aboveskip=-\baselineskip]
(*CHECK*)
package A
    Real x=1;
    model M
        Real x=2 if (...);
        /* ? x */ /* x dependes on a condition? */
    end M;
end A;
\end{lstlisting}
\end{quote}

(* (?) Check if an inner is affected by its condition. *)

%%%%%%%%%%%%%%%%

\subsection{Class names}

Names of classes are almost insignificant after collecting elements of
main/base classes.  Records are exceptions which are refered to by
names at runtime.

Classes of packages/instances can be nameless if modifiers are applied
but not assigned by a name.  However, dotted forms of variable
references can be used to refer to packages/instances (like "x.y.P"),
because only used classes need to be referred to.  In addition, each
class (a main/base part of a class) is generally associated to three
names: $\bullet$\,a name given to a lexically appearing definition
body, $\bullet$\,a name lastly assigned by a short_class_definition,
and $\bullet$\,a name indicating where the definition body is taken
from.  The second name by a short_class_definition is the name of a
usual sense and needed to refer to a record name (needed in
instances).  The third name indicating its origin is needed to obtain
an enclosing class (needed in packages).

%%%%%%%%%%%%%%%%

\subsection{Class_equality}\label{class-equality}

Generally, an equality or identity of a class is not significant, but
class_compatibility is used, instead.  An identity of a class is not
important even for records, which would be meaningful since they are
used in function calls or copying in algorithms.  See
Fact~\ref{record-class-identity} for records.  In most cases,
class_equality is treated as a textual equality, which is required
when removing duplicate class definitions.

$\mathbf{input}$ and $\mathbf{outer}$ are not a part of a type.

\begin{qquote} (\specref{4.5.1}\/) A base-prefix applied in the
short-class definition does not influence its type, but is applied to
components declared of this type or types derived from it; see also
section 4.5.2.\end{qquote}

(?) (*CHECK*) The equality of functions is by textual (defined?).  The
overloading resolution is based on a set of functions (requiring
uniqueness), thus, the equality of functions needs be defined.

%%%%%%%%%%%%%%%%

\subsection{(Class_compatibility)}\label{class-compatibility}

A redefinition/redeclaration can be applied to compatible classes.

\subsubsection*{Compatibility}

\specref{6.3}\/ defines compatibility, which applies when a class is
redeclared.  It is APPROXIMATELY summarized here.  The conditions of
$A<:B$, $A$ is a subtype of $B$.

\begin{qquote}\label{quo:class-compatibility} (\specref{6.3}\/) A is a
subtype of the type of B, iff [intuitively all important elements of B
must be present in A].\end{qquote}

\begin{itemize}

\item A extends C iff B extends C, where C is an operator record.

\item $A$ extends $C_A$ iff $B$ extends $C_B$, where $C_A$ and $C_B$
are ExternalObject with the same name.

\item If $B$ is non-replaceable, $A$ must be non-replaceable.  (* Why
negatively expressed? *)

\item If $B$ is transitively non-replaceable, $A$ must be transitively
non-replaceable.

\item $\forall C_B \in B, \exists C_A \in A,\; C_A <: C_B$ where $C_B$
and $C_A$ are components of the same name and visibility in each
corresponding class.

\item If $B$ is replaceable, OK.

\item If $B$ is neither transitively non-replaceable nor replaceable,
$A$ must be linked (?) to the same class ((?) linked).

\item If $B$ is a non-replaceable long class definition, $A$ must be a
long class definition.

\item Analogical (flow, effort (default), or stream) should be the
same.

\item $\mathrm{variability} A \leq \mathrm{variability} B$, with
respect to the variability ordering.

\item Prefixes input and output must be the same.

\item Prefixes inner, outer, and inner-outer must be the same.

\item If $B$ is final, $A$ must be final and has semantically the same
value.  (This rule does not apply to redeclarations.)

\item The number of array dimensions in $A$ and $B$ must be the same.

\item Conditional_components are only compatible with conditional
components, where the conditions are semantically the same.

\item Class_kinds must satisfy kinds compatibility.

\item An enumeration is only compatible with an enumeration.  A
compatible enumeration (except for enumeration(:)) must have the same
literals in the same order.  Or, $\forall A,\; A <:
\mathit{enumeration}\,(:)$.

\item $B$ is a built-in iff $A$ is a built-in and $A=B$.

declared variability???

\end{itemize}

\paragraph{Supplementary notes}

\begin{itemize}

\item Plug-compatibility is stricter than compatibility.

\item Compatibility functions is stricter than compatibility.

\item A modifier must be compatible or plug-compatible with the
element being modified.

\item $A$ is an operator record when it extends an operator record.

\item $A$ is an ExternalObject when it extends an ExternalObject.

\end{itemize}

\begin{QRULE} (?) The conditions on prefixes are checked before
inheriting the prefixes from a replaceable to a
redeclaration.\end{QRULE}

\subsubsection*{Plug-compatibility}

Plug-compatibility is defined in \specref{6.4}.

\subsubsection*{Function compatibility}

Function compatibility is defined in \specref{6.5}.

%%%%%%%%%%%%%%%%

\subsection{Class_kinds}

The table in \specref{4.6}\/ shows the class_kinds.  Every kind is
syntactically the same as a class, but some restrictions on its
contents apply to the class with regard to its kind.

\subsubsection*{Class_kinds}\label{sec:class-kinds}

The following is the copy of the table in \specref{4.6}.

\begin{quote}
\begin{description}

\item[model] is a general collection of elements.

\item[class] is a restricted collection that contains extends-clauses,
class definitions, and annotations.  Note that it was a synonym of
$\mathbf{model}$ by the old definition, and some_implementations
accept other elements.  See Quote~\ref{quo:class-is-restricted}.

\item[block] is a restricted {model}, whose connectors are directed by
$\mathbf{input}$/$\mathbf{output}$.

\item[package] contains similar elements as $\mathbf{record}$ that is
restricted to contain constants/parameters but no variables.
(*CHECK*) It can contain protected sections.

\item[type] is one of the simple_types, arrays of a type, or one
extending a type.

\item[record] is a record in a usual sense.  A component of a record
is a type, a record, or an array of them.  It contains no
equation/algorithm sections nor protected sections (but it can contain
functions).

\item[operator_record] is a record, but is restricted for
operator-overloading.  It can contain the same elements as records,
but also it can contain operators.

\item[operator] contains similar elements as $\mathbf{package}$, that
is restricted to contain only functions as definitions.  An operator
is only defined in an operator_record.

\item[(operator_function)] is $\mathbf{operator}$ with one function.

\item[connector] is either a {type} or a {record}, but it is allowed
to connect on it.  It is actually a subkind of them.  The use of a
keyword $\mathbf{connector}$ defines a record, when it is used in a
long class definition.  It gives a connector-status to a type/record,
when it is used in a short class definition.  Connector declarations
are implicit in expandable_connectors.

\item[expandable_connector] is a connector.

\item[function] is a function in a usual sense.  It is syntactically a
package (not an instance).  It can contain either one algorithm
section or one external interface.  It does not contain equation
sections nor initial algorithm sections.  It can contain component
declarations (variables) of $\mathbf{type}$, $\mathbf{record}$,
$\mathbf{operator~record}$, and $\mathbf{function}$.  Only functions
can contain an external interface.  See (\specref{12.2}\/).  (?)
Functions can contain only input/output variables.

\end{description}
\end{quote}

(* $\mathbf{function}$ is not stated as a restriction of
$\mathbf{package}$ while $\mathbf{operator~function}$ is. *)

\subsubsection*{Kind restrictions}

The contents of a class are: import-clauses, extends-clauses, class
definitions (class/enumeration/der), component declarations
(variable/parameter/constant), equation/algorithm sections
(with/without initial), and external interfaces.  Any class can
contain import-clauses, extends-clauses, and other class definitions.
The other contents are restricted by its kind.  In summary, the
relations of content restrictions are:
%%%%
\vskip 1ex
\begin{tabular}{|l||l*{3}{|p{4.5em}}|l|}
\hline
kind &variable &{parameter constant} & equation &algorithm &{similar kinds} \\
\hline
type & & & & & \\
model &\checkmark &\checkmark &\checkmark &\checkmark & block, (class) \\
class & & & & & \\
record &\checkmark &\checkmark & & & operator_record \\
package & &\checkmark & & & \\
connector &\checkmark &\checkmark & & & expandable_connector \\
operator &\checkmark &\checkmark & & & operator_function \\
function & & & &\checkmark & \\
\hline
\end{tabular}
\vskip 1ex
%%%%

%% (?) Functions can contain only input/output variables.  But, impure
%% functions may refer to variables in enclosing classes.

\begin{QFACT}\label{kind0} (?) The classes that contain state
variables are: ($\mathbf{class}$), $\mathbf{model}$, $\mathbf{block}$,
$\mathbf{record}$, $\mathbf{operator~record}$, $\mathbf{connector}$,
$\mathbf{expandable~connector}$, $\mathbf{function}$.  Conversely, the
classes that do not contain states are: $\mathbf{package}$,
$\mathbf{operator}$, and $\mathbf{operator~function}$.\end{QFACT}

\subsubsection*{Extending and class_kinds}

The table in \specref{7.1.3}\/ shows the restrictions on derived vs.\@
base class relations.  It shows that one class can extend some class.
$A$ can extend $B$, if $A\in\mathit{kind}_A$ and $B\in\mathit{kind}_B$
and $\mathit{kind}_B<:_k\mathit{kind}_A$ (note it is swapped).
%%%%
\begin{itemize}\parsep=0pt\itemsep=0em
\item $\mathbf{class}$ $<:_k$ {\it{}any\/}
%%%%
\item $\mathbf{block}$ $<:_k$ $\mathbf{model}$
\item $\mathbf{record}$ $<:_k$ $\mathbf{model}$ (redundant)
%%%%
\item $\mathbf{record}$ $<:_k$ $\mathbf{block}$
%%%%
\item $\mathbf{type}$ $<:_k$ $\mathbf{connector}$
\item $\mathbf{record}$  $<:_k$ $\mathbf{connector}$
\item $\mathbf{operator~record}$ $<:_k$ $\mathbf{connector}$
\item $\mathbf{expandable~connector}$ $<:_k$ $\mathbf{connector}$
%%%%
\item $\mathbf{function}$ $<:_k$ $\mathbf{operator~function}$
\end{itemize}
%%%%
The excerpt is the entries that are highlighted by shading in the
table.  We omit the reflexive relation on the same kind:
$\mathit{kind}_A<:_k\mathit{kind}_A$.

\subsubsection*{(?) Kind compatibilty}

(?) It is unsure that the extending relation is a part of the
subtyping relation.  See Section~\ref{class-compatibility} for the
subtyping relation.  (?) A kind is not compatible (accepting
redeclarations) to other kinds.

\subsubsection*{Kind restrictions in short_class_definitions}

\begin{QFACT}\label{kind-restriction} Short_class_definitions follow
the base class restrictions (in \specref{7.1.3}\/).  For example, it
is legal to define a boolean as a connector.
"$\mathbf{connector}\;\mathit{BooleanInput}=\mathbf{input}\;\mathbf{Boolean}$".
\end{QFACT}

It is a consequnce of the description of rewriting a
short_class_definition to a class definition with an extends-clause:
%%%%
\begin{qquote}\label{quo:short-class-definitions} (\specref{4.5.1}\/) A
class definition of the form ... is identical ... to the longer
form.\end{qquote}

\subsubsection*{The class kind}

\paragraph{Remark}

The $\mathbf{class}$ kind is defined to be more restricted than in
\specref{7.1.3}.
%%%%
\begin{qquote}\label{quo:class-is-restricted} (\specref{7.1.3}\/) A class
may only contain class-definitions, annotations, and extends-clauses
(having any other contents is deprecated).\end{qquote}
%%%%
By the deprecated definition, it is notable that a more restricted
kind can be extend by $\mathbf{class}$.
%%%%
\begin{QFACT}\label{kind1} It is required to check base classes
respect the restriction.\end{QFACT}

%%%%%%%%%%%%%%%%

\subsection{Predefined_classes}

Predefined_classes are called predefined_types in the_specification,
but some are not types.

\subsubsection*{Simple_types}

A simple_type is one of {Real}, {Integer}, {Boolean}, {String}, and
$\mathit{enumeration}$.  Enumerations are types.  And, thus, an
enumeration is defined by a short_class_definition using the
$\mathbf{type}$ class kind.  The names {Real}, {Integer}, {Boolean},
and {String} are not keywords but cannot be redefined.

\begin{qquote}\label{quo:reserved-words} (\specref{4.8}\/) The names Real,
Integer, Boolean and String are reserved such that it is illegal to
declare an element with these names.\end{qquote}

Each simple_type has modifiable attributes and thus actually
represents a set of types.  Variable declarations have effective
attributes of simple_types, but values of expressions have no
attributes, and their values are primitive_types like usual reals or
integers.  See Section~\ref{sec:attributes}.

There are some predefined conversion functions among the simple_types
(not for all pairs).
$\mathrm{Integer}:\mathit{enumeration}\rightarrow\mathrm{Integer}$,
$\mathit{enumeration}:\mathrm{Integer}\rightarrow\mathit{enumeration}$

\paragraph{Remark}

There are no predefined conversion functions between Boolean and
Integer.  However, array binding with these index types assumes
mapping between these types, such as false=1 and true=2, etc.

Note that there are no constructors to the simple_types.  See
Fact~\ref{constructors-to-predefined-types}.

The least required ranges of the domains of types are defined by
Modelica.Constants.inf for Real, and Modelica.Constants.Integer_inf
for Integer.  (?) The behavior on overflows is unspecified.

\subsubsection*{Integer, String, enumerations}

\begin{QFACT}\label{simple-type-conversions} The class names of
$\mathrm{Integer}$, $\mathrm{String}$, and $\mathit{enumeration}$ can
be used as conversion functions.  Since there is a single namespace
for classes and functions, these names are overloaded.\end{QFACT}
%%%%
See \specref{3.7.1}.  $\mathrm{Integer}$ and names of enumerations
convert between integers and enumerations.

An enumeration definition by $\mathbf{enumeration}\,(:)$ is an
unspecified list, which must be redeclared (\specref{4.8.5.4}\/).

\subsubsection*{Ordering of simple_types}

See relational operators in Section~\ref{sec:relational-operators}.

\subsubsection*{Predefined_classes}

Predefined_type is simple_types and other classes defined in
\specref{4.8.8}.  They are {Clock}, {StateSelect}, {ExternalObject},
{AssertionLevel}, {Connections}, and classes for graphical
annotations.

\begin{description}

\item [Clock] is a class.

\item [StateSelect] is an enumeration.

\item [ExternalObject] is a partial class.

\item [AssertionLevel] is an enumeration.

\item [Connections] is a package.

\end{description}

\subsubsection*{Connections}

Connections is a package is used to remove overdetermined equations.
It is described in \specref{9.4}.  A type/record defining
$\mathit{equalityConstraint}\,(x,y)$ is called an
overdetermined_type/record.  Connections works with
overdetermined_types/records.  It contains equation_operators
$\mathit{branch}$, $\mathit{root}$, and $\mathit{potentialRoot}$
(appear as equations), and boolean-valued operators $\mathit{isRoot}$
and $\mathit{rooted}$ (appear as functions).

Connecting connectors may induce an undirected graph of
overdetermined_types/records in connectors as nodes and equality as
edges.  Such a graph may have cycles when the equations are
overdetermined.  A graph is reduced to trees by removing some of the
edges by replacing them with weaker contraints.  Operators in
Connections are used to specify trees in the graph.

$\mathit{branch}\,(m.c.r,m.c.r)$ makes an edge between
overdetermined_types/records.  $\mathit{root}\,(m.c.r)$ specifies a root
of a tree of overdetermined_types/records.
$\mathbf{connect}\,(m.c,m.c)$ creates weak edges for the
overdetermined_types/records in the connectors.  Some weak edges may
be marked (as removed), and the equality of marked edges are replaced
by $\mathit{equalityConstraint}\,(x,y)$ equations.

%%%%%%%%%%%%%%%%

\subsection{Records}

A record is a record in a usual sense.  Restrictions on components of
a record are described in Section~\ref{sec:class-kinds}.

\subsubsection*{Class_equality}

\begin{QFACT}\label{record-class-identity} An identity of a record
class is not clearly defined.  The equality of the lists of the
defined names may be required.\end{QFACT}

Some_implementations may accept the following code:
%%%%
\begin{quote}
\begin{lstlisting}[aboveskip=-\baselineskip]
record R0
    Real v;
end R0;

record R1
    Integer v;
end R1;

model M
    R1 x1 (v=20);
    R0 x0 = x1;
end M;
\end{lstlisting}
\end{quote}

See also Section~\ref{class-equality}.  (*CHECK*) Function calls may
require stricter equality.

\subsubsection*{Casting}

A record constructor can take a class instance, and it is called
casting.  See \specref{12.6.1}.

\subsubsection*{equalityConstraint}

A function $\mathit{equalityConstraint}$ can be defined in types and
records.  $\mathit{equalityConstraint(...)}=0$ is added as equations,
when it is defined (0 in the RHS is used for an array of zeros).
See \specref{9.4.1}.

%%%%%%%%%%%%%%%%

\subsection{Arrays}

\subsubsection*{Component array declarations}

\begin{QFACT} Dimensions of component arrays are translation-time
constants, while dimensions of other arrays (appearing in functions)
can be runtime values.\end{QFACT}

\begin{qquote} (\specref{4.4.2}\/) Array dimensions shall be
non-negative parameter expressions, or the colon operator\end{qquote}

\begin{qquote} (\specref{10}\/) The number of dimensions of an array
is fixed and cannot be changed at run-time...  However, the sizes of
array dimensions can be computed at run-time\end{qquote}

An extent in an array dimension is non-negative, that is, it can be
zero.  Those variables with size zero, obviously, cannot be used.

Constants in array dimensions are fairly arbitrary expressions.  They
can refer to other components.  But, constant-folding may fail, when a
dimension depends on a value in the array, which directly or
indirectly causes cycles in the dependency.  For example,
$\mathit{dim}$ in $x[\mathit{dim}]$ may not refer to a constant
$x[1].c$.

\subsubsection*{Array dimension ordering}

\begin{QFACT} Multi-dimensional arrays are arrays of arrays.  The
ordering of an array dimension is "row-major".\end{QFACT}

See code examples in (\specref{10.1}\/):
%%%%
\begin{qquote} (\specref{10.1}\/) It is possible to mix the two
declaration forms although it might be confusing.
\begin{lstlisting}[]
Real[3,2] x[4,5]; // x has type Real[4,5,3,2];
\end{lstlisting}
\end{qquote}
%%%%
Note that accessing the example array by $x[\textrm{-},\textrm{-}]$
returns an array of $[3,2]$, which means, partial indexing is applied
from the left in the dimension.  This ordering is row-major, despite
that Fortran is column-major.

The definition of the {fill} operator by recursion also implies the
row-major ordering (\specref{10.3.3}\/).

\subsubsection*{Rectangle-ness of array dimension}

An array is rectangular, and an array of arrays has the same size in
each element.  This applies to the uses of nested array constructors.

\subsubsection*{Non-integer array dimension}

Enumeration types and Boolean can be used as a dimension, and their
elements are used in indexing.  They can also be used as ranges in
for-loops.  See code examples in (\specref{10.1}\/):
%%%%
\begin{quote}
\begin{lstlisting}[aboveskip=-\baselineskip]
type TwoEnums = enumeration(one,two);
Real[TwoEnums] y;
\end{lstlisting}
\end{quote}

An index of enumeration types is ordered by the declaration order.  An
index of Boolean is ordered as $\mathit{false}<\mathit{true}$.
While indexing by different types is illegal, arrays are bound or
assigned without regard to the index types.

\begin{qquote} (\specref{10.1}\/) Binding equations of the form x1 =
x2 as well as declaration assignments of the form x1 := x2 are allowed
for arrays independent of whether the index types of dimensions are
subtypes of Integer, Boolean, or enumeration types\end{qquote}

\begin{qquote} (\specref{10.1.1}\/) $\bullet$\,An array dimension
indexed by Boolean values has the lower bound false and the upper
bound true.  $\bullet$\,An array dimension indexed by enumeration
values of the type E=enumeration(e1, e2, ..., en) has the lower bound
E.e1 and the upper bound E.en.\end{qquote}

\subsubsection*{Function promotions to arrays}

Functions are promoted to array arugments.  In promoting functions,
array arguments are promoted to higher dimensinal arrays.

\subsubsection*{Redeclaring array dimensions}

Array dimensions may be redeclared.
%%%%
\begin{qquote} (\specref{7.3.3}\/) Modification on array dimensions
are redeclarations.\end{qquote}

\subsubsection*{Empty arrays}

\begin{QFACT} An array indexed with
$\mathit{lower~bound}>\mathit{upper~bound}$, such as $x[1{:}0]$, is an
empty array.\end{QFACT}

See \specref{10.7}\/ for empty arrays.  This may actually appear
(under some condition) in Interfaces.PartialMedium.BaseProperties (in
Media/package.mo).

\begin{quote}
\begin{lstlisting}[aboveskip=-\baselineskip]
    replaceable partial model BaseProperties
    ...
    equation
      if standardOrderComponents then
        Xi = X[1:nXi];
\end{lstlisting}
\end{quote}

A {fill} with a zero dimension can be used to create an empty array,
such as $\mathrm{fill}\,("",0)$.

\subsubsection*{Homogeneity of array elements}

Instances of an array elements are similar, in that, the classes of
the elements are similar, and the number of dimenions of them are the
same, because redeclarations (in-element or in-modifier) cannot depend
on an index, although the values of constants can be different, and
the dimesion of arrays can be different.  The classes may have
conditional_components, but they are of similar classes even though
they can be disabled.  This ensures the identity of promoting
operators to arrays.  (?) Class similarity is defined by class
compatibility.

\subsubsection*{(?) Array class definition}

An array type definition for simple types
$\mathbf{type}\;B=\mathrm{Real}\;[4]$ is legal.  (?) In contrast, a
definition like $\mathbf{model}\;B=A\;[4]$ does not work.

\subsubsection*{Array package definition}

\begin{QFACT} An array package definition like
$\mathbf{package}\;B=A\;[4]$ is illegal.\end{QFACT}
%%%%
Some_implementations simply ignore subscripts, that is, $B$ is an
alias of $A$.

%%%%%%%%%%%%%%%%

\subsection{Array expressions}

\subsubsection*{Array indexing}

\begin{QFACT} Array indexing is only in
$\mathit{identifier}\,[i0,i1,...]$.  General forms
$\mathit{expression}\,[i0,i1,...]$ are not allowed.\end{QFACT}

\subsubsection*{Array ranges}\label{sec:array-ranges}

$\textrm{-}:\textrm{-}$ and $\textrm{-}:\textrm{-}:\textrm{-}$ create
a one dimensional array (vector) of Real, Integer, Boolean, and
enumerators.  For numbers, they create Integer vectors if all elements
are Integers, or Real vectors otherwise.  Triple argument ones are not
defined for Boolean and enumerations.  They may create an empty
vector.  See \specref{10.4.3}\/ and also \specref{3.2}.

\subsubsection*{Array constructors}

An expression $\{...\}$ creates an array, which is one-dimensional if
the elements are scalar or multidimensional otherwise.  An array
constructor cannot be empty by the syntax definition, although an
array can have zero in its dimension (it ensures the type an array
constructor returns).

\subsubsection*{Predefined functions on arrays}\label{sec:array-size}

$\mathrm{ndims}\,(A)=0$ and $\mathrm{size}\,(A)=\{\}$ (an empty array)
for a scalar $A$ (See \specref{10.3.1}\/).

Functions are defined to create certain arrays: {identity},
{diagonal}, {zeros}, {ones}, {fill}, and {linspace}
(See \specref{10.3.3}\/).  Note that {identity}, {zeros} and {ones}
return an integer array, and {linspace} returns a real array.

Some can be easily defined by others.
\begin{eqnarray*}
&& \mathrm{identity}\,(n)=\mathrm{diagonal}\,(\mathrm{ones}\,(n))\;,\\
&& \mathrm{zeros}\,(...)=\mathrm{fill}\,(0,...)\;,\\
&& \mathrm{ones}\,(...)=\mathrm{fill}\,(1,...)\;,\\
&& \mathrm{linspace}\,(\mathit{lb},\mathit{ub},n)
=\mathit{lb}:\mathit{ub}:(\mathit{ub}-\mathit{lb}/(n-1))\;, \textrm{and}\\
&& \mathrm{fill}\,(e,n_0,...)=\mathrm{fill}\,(\mathrm{fill}\,(e,...),n_0)\;.
\end{eqnarray*}

%%%%%%%%%%%%%%%%

\subsection{Iterators of for}\label{sec:for-iterator}

Iterators are used in for-equations, for-statements, reductions, and
comprehensions.  An iterator ranges in Real, Integer, Boolean, and an
enumeration, and its range is specified by a vector expression or by a
type of Boolean or an enumeration.

\subsubsection*{Range expressions of iterators}

For-iterators with multiple iterators are defined by expanding to
nested ones.
%%%%
\begin{qquote} (\specref{11.2.2.3}\/) it can be expanded into the
usual form by replacing each "," by 'loop for' and adding extra 'end
for'.\end{qquote}

Range expressions of for-equations are translation-time constants, and
they can include iterators in the nested ranges.

\begin{qquote} (\specref{8.3.2}\/) The expression of a for-equation
shall be a parameter expression.\end{qquote}

It is allowed to refer to an iterator in vector expressions in a
single for-equation.
%%%%
\begin{quote}
\begin{lstlisting}[aboveskip=-\baselineskip]
equation
for i in 1:s, j in 1:i-1 loop
    x[i,j]=j;
end for;
\end{lstlisting}
\end{quote}

The following quotes of the scope description might conflict with the
handling by nesting iterators.

\begin{qquote}\label{quo:iterator-scope} (\specref{8.3.2}\/) It is
evaluated once for each for-equation, and is evaluated in the scope
immediately enclosing the for-equation.\end{qquote}

\subsubsection*{Implicit iterator ranges}\label{sec:implicit-iterator-ranges}

A range expression of an iterator is optional, and the range is
inferred from the uses of an iterator.  Implicit ranges are accepted
for arrays with dimensions of Integer, Boolean, and enumeration.
Implicit iterator ranges are allowed in any {\it{}for}-iteration in
equations, statements, reductions, and array constructors.

It is assumed that the range of an implicit iterator is inferred by
the range of the subscript where an iterator occurs.

See the example in \specref{4.8.5}.
%%%%
\begin{quote}
\begin{lstlisting}[aboveskip=-\baselineskip]
/* 4.8.5 */
type DigitalCurrentChoices = enumeration(zero, one);
Real x[DigitalCurrentChoices];
for e loop
    x[e] := 0.;
end for;
\end{lstlisting}
\end{quote}

See the example in \specref{11.2.2.3}.
%%%%
\begin{quote}
\begin{lstlisting}[aboveskip=-\baselineskip]
/* 11.2.2.3 */
Real x[4,3];
algorithm
for j, i in 1:2 loop
// The loop-variable j takes the values 1,2,3,4 (due to use)
// The loop-variable i takes the values 1,2 (given range)
x[j,i] := j+i;
end for;
\end{lstlisting}
\end{quote}

\begin{QFACT} Boolean and enumerations can be an index space for
arrays.  However, Interger, even with min/max attributes specified,
cannot be an index space.\end{QFACT}

\begin{quote}
\begin{lstlisting}[aboveskip=-\baselineskip]
/*ILLEGAL*/
type R = Integer(min=3,max=5);
Real x[R];
\end{lstlisting}
\end{quote}

%%%%%%%%%%%%%%%%

\subsection{Overloaded operators}

Operators can be overloaded on the arguments, one of which is an
operator record or an array of it.

\subsubsection*{Overloading}

\begin{QFACT} Functions are not overloaded.  Functions are classes and
having the same name is a duplication.\end{QFACT}

Some predefined functions are overloaded.  They include simple_type
names of Integer, String, and $\mathit{enumeration}$ (see
Fact~\ref{simple-type-conversions}).  They also include ones taking
array arguments of varying dimensions, including ndims, size, and
reductions (min, max, sum, and product).  Also note that, functions
which take indefinite number of arguments cannot be defined as
functions.

\subsubsection*{List of operators}

The operators can be overloaded.  See \specref{14}.  The set of the
overloaded operators is fixed.

\begin{quote}
\spaceskip=1.5em
\begin{verbatim}
- not
+ - * / ^ == <> > < >= <= and or
constructor 0
String
\end{verbatim}
\end{quote}
%%%%
The \verb'-' appears twice as a unary and a binary.  Note that the
list lacks unary \verb'+', becuase it is not overloaded.

\begin{QFACT} Operators except constructors need be defined in an
operator record on which the operators take the arguments.
Constructors need be defined in an operator record on which the
constructors return the value.  Binary operators can be defined in one
of the left or right operands.  (?) If functions to an operator are
defined in both the left and right operands, they should be equal
(equality of functions may be textual).\end{QFACT}

This rule is implied by the matching rules in \specref{14.4}\/ and
\specref{14.5}.

\begin{QFACT}\label{constructors-to-predefined-types} Constructors are
not defined on the simple_types.\end{QFACT}

\begin{quote}
\begin{lstlisting}[aboveskip=-\baselineskip]
/*ILLEGAL*/
Real x = Real(10);
\end{lstlisting}
\end{quote}

\subsubsection*{Resolution rules}

The rules of resolving overloading are described in \specref{14.4}\/
and \specref{14.5}.

The application of an unary operator $\oplus a$ with $a$ of class $A$
is determined as follows.  The tests are tried in the specified order.
$\tilde{A}.\oplus$ denotes a set of functions of the operator $\oplus$
in the operator_record $\tilde{A}$ with arrays stripped off from $A$.

\begin{itemize}

\item If $A$ is a simple_type or an array of it, apply a predefined
operator.

\item Apply a function $f \in \tilde{A}.\oplus$, whose argument type
is $A$.  If $f$ is not unique, it is illegal.

\item Apply a function $f$ by promoting it to an array argument.

\end{itemize}

The application of a binary operator $a \oplus b$ with $a$ of class
$A$ and $b$ of class $B$ is determined as follows.  The tests are
tried in the specified order.

\begin{itemize}

\item If $A$ and $B$ are simple_types or arrays of them, apply a
predefined operator.

\item Apply a function $f \in \tilde{A}.\oplus$ or $f \in
\tilde{B}.\oplus$, whose argument types are $A\times B$.  If $f$ is
not unique, it is illegal.

\item Apply $f \in \tilde{A}.\oplus$ with trying type conversions applicable on
the second argument.

\item Apply $f \in \tilde{B}.\oplus$ with trying type conversions
applicable on the first argument.

\item Apply a function $f$ by promoting it to array arguments.

\end{itemize}

\subsubsection*{Operators}

The names of the operators need be quoted by {\tt{}'} to make them
recognized as ordinary identifiers, for example, \verb|'+'|.  (?) It
is not clear quoting is redundant for {\tt{}'constructor'} and
{\tt{}'String'}.

Refer to sections, {14.2} for constructors, {14.3} for conversions,
{14.4} for binaries, and {14.5} for unaries.

\subsubsection*{Operators on arrays}

Arrays are the general entity including scalars, vectors and matrices.
Scalars are arrays with no dimensions, vectors with one dimension, and
matrices with two dimensions.

\begin{QFACT} Overloaded operators on scalars are promoted to
arrays.\end{QFACT}

\begin{qquote} (\specref{14}\/) The operator or operator function
must be encapsulated;\end{qquote}

\begin{qquote} (\specref{14}\/) It is not legal to extend from an
operator record, except as a short class definition modifying the
default attributes...\end{qquote}

\subsubsection*{(?) Default constructors}

%%%%%%%%%%%%%%%%

\subsection{Predefined operators}\label{sec:predefined-operators}

Operators are overloaded on the arguments of a simple_type or an array
of it.  The set of operators is different from the one for operator
records.

\subsubsection*{List of operators}

\begin{quote}
\spaceskip=1.5em
\begin{verbatim}
+ - .+ .- not
= + - * / ^ .+ .- .* ./ .^ and or
== <> > < >= <=
\end{verbatim}
\end{quote}

\subsubsection*{Equalities}

\paragraph{(?) Equality {\tt{}=}}

The equality and assignment are element-wise.  Classes which admit
equality are simple_types or records whose components admit equality.
Arrays admit equality when the class of the elements admits equality.

\subsubsection*{Operators on numbers}

The numbers (or the numerical types) are subtypes of either Real or
Integer.  Conversion from an Integer to a Real is applied, when an
Integer expression is passed to an operator requiring a Real.  The
usual types for the arithmetic operators ({\tt{}+} {\tt{}-} {\tt{}*}
{\tt{}/}) are assumed: $\mathbb{Z} \times \mathbb{Z} \rightarrow
\mathbb{Z}$ and $\mathbb{R} \times \mathbb{R} \rightarrow \mathbb{R}$.
The type of the operator {\tt{}\char`\^} is $\mathbb{R} \times
\mathbb{R} \rightarrow \mathbb{R}$.

Note that subtypes are mentioned for Real and Integer, because Real
and Integer are defined as $\mathit{type}$ and thus it is possible to
extend them.

\begin{qquote} (\specref{10.6}\/) In all contexts that require an
expression which is a subtype of Real, an expression which is a
subtype of Integer can also be used; the Integer expression is
automatically converted to Real.\end{qquote}

Operators on numbers and strings are defined for arrays in
\specref{10.6}.

%% {10.6}{Scalar, Vector, Matrix, and Array Operator Functions}
%% {14.6}{Example of Overloading for Complex Numbers}

\paragraph{Unary {\tt{}+} {\tt{}-} {\tt{}.+} {\tt{}.-}}

The operators \verb'+' and \verb'.+' on numbers are the identity and
element-wise.  The operators \verb'-' and \verb'.-' on numbers are
negation and element-wise.  These are the same with/without a dot.

\paragraph{Binary {\tt{}+} {\tt{}-}}

The operators \verb'+' and \verb'-' on numbers are the addition and
the subtraction, and element-wise.

\paragraph{Binary {\tt{}.+} {\tt{}.-}}

The operators \verb'.+' and \verb'.-' on numbers are similar to
\verb'+' and \verb'-'.  The difference to ones without a dot is that
if one (not both) of the arguments is a scalar, it is promoted to an
array.

\paragraph{Binary {\tt{}*}}

The operator \verb'*' on numbers is the multiplication.  If one of the
arguments is a scalar, it is the scalar product.  1-dimensional
(vector) and 2-dimensional (matrix) cases are, as usual, an inner
product, a matrix-vector product, or a matrix-matrix product.  Higher
dimensional multiplication is illegal.

\paragraph{Binary {\tt{}.*}}

The operator \verb'.*' on numbers is the element-wise multiplication.
If one of the arguments is a scalar, it is promoted to an array, that
is, it is a scalar product.

\paragraph{Binary {\tt{}/}}

The operator \verb'/' on numbers is the element-wise division.  The
RHS should be a scalar.

\paragraph{Binary {\tt{}./}}

The operator \verb'/' on numbers is the element-wise division.  If one
of the arguments is a scalar, it is promoted to an array.

\paragraph{Binary {\tt{}\char`\^}}

The operator \verb'^' on numbers is the exponentiation.  The arguments
are a scalar-scalar pair or a matrix-scalar pair.  If the arguments
are scalar-scalar, arguments of Integers are promoted to Reals.  If
the arguments are matrix-scalar, it is the matrix exponentiation where
the scalar argument must be a non-negative integer.

\paragraph{Binary {\tt{}.\char`\^}}

The operator \verb'.^' on numbers is the element-wise exponentiation.
If one of the arguments is a scalar, it is promoted to an array.
Arguments of Integers are promoted to Reals.

\subsubsection*{Operators on booleans}

\paragraph{Booleans {\tt{}and} {\tt{}or} {\tt{}not}}

The operators work on boolean values and element-wise.

\subsubsection*{Operators on strings}

\paragraph{Binary {\tt{}+}}

The operator \verb'+' on strings is the concatenation and
element-wise.

\subsubsection*{Relational operators}\label{sec:relational-operators}

\paragraph{Relationals {\tt{}==} {\tt{}<>} {\tt{}<} {\tt{}<=} {\tt{}>}
{\tt{}>=}}

Relational operators are defined on the scalar simple_types only, but
not for arrays.  Moreover, equality on class instances is not defined.

\begin{qquote} (\specref{3.5}\/) Relational operators ... are only
defined for scalar operands of simple types.\end{qquote}

The ordering of elements of simple_types is defined in \specref{3.5}.
On numbers, the arguments may be promoted from {Integer} to {Real}, if
the types of the arguments differ.  Boolean is ordered as
$\mathit{false}<\mathit{true}$.  A set of enumerators is ordered by
the definition order.  The ordering in String is defined by the libc
function $\mathit{strcmp}\,(x,y) \bowtie 0$ (replacing $\bowtie$ with a
relational operator).

\paragraph{Remark}

The predefined operators are defined for the simple_types, although
\specref{14.3}\/ and \specref{14.4}\/ refer to the predefined_types.
The predefined_types excluding the simple_types cannot have operators,
because they are not operator records.

\paragraph{Remark}

$\mathit{strcmp}$ compares strings as unsigned bytes (not unicode
strings).

%%%%%%%%%%%%%%%%

\subsection{Type conversions}

Function argument conversions are defined in \specref{12.4.6}.

\begin{QFACT}\label{no-cycle-in-conversions} Type conversions are
defined by constructors.  Conversions may be chained, but it is
illegal that they make a cycle.\end{QFACT}

\begin{qquote} (\specref{14.2}\/) For a pair of operator record
classes C and D and components c and d of these classes both of
C.'constructor' (d) and D.'constructor' (c) shall not both be
legal.\end{qquote}

%%%%%%%%%%%%%%%%

\subsection{Constant expressions}

A dimension of a component array needs to be translation-time
constants, and thus, folding constants is essentially required.  A
vast set of expressions are allowed for array dimensions, and
especially for example, they include array indexing because it is an
operator.  They include user defined functions, too.  Note that we
refer here by constants to both constants and parameters (parameter is
a wider set in the_specification).

\begin{qquote} (\specref{3.8.2}\/) a function or operator with
parameter subexpressions is a parameter expression.\end{qquote}

\specref{3.8.1}\/ and \specref{3.8.2}\/ list constant expressions
(functions and variables) which can take non-constant arguments:
%%%%
\begin{quote}\spaceskip=2em
initial
terminal
der
edge
change
sample
pre
\end{quote}
%%%%
\begin{quote}\spaceskip=2em\let~=\enspace
ndims(A)
cardinality(c)
end
size(A) size(A,~j)
Connections.isRoot(A.R)
Connections.rooted(A.R)
\end{quote}

%%%%%%%%%%%%%%%%

\subsection{Attributes}\label{sec:attributes}

The attributes are parameters predefined in the simple_types.  They
can be modified by names through modifiers, but cannot be accessed by
names using a dot notation.  They can only be accessed in a subclass
of simple_types.  The $\mathit{value}$ attribute can be accessed by
referencing a variable.

\begin{qquote}\label{quo:attribute-access} (\specref{4.8}\/)
Attributes cannot be accessed using dot notation, and are not
constrained by equations and algorithm sections. E.g. in Real
x(unit="kg")=y; only the values of x and y are declared to be equal,
but not their unit attributes, nor any other attribute of x and
y.\end{qquote}

A dot notation is used for enumerations when accessing an enumerator
element.

%%%%%%%%%%%%%%%%

\subsection{Expressions}

\subsubsection*{(?) pure}\label{pure-expression}

$\mathrm{pure}\,(...)$ is not well specified.
%%%%
\begin{qquote} (\specref{12.3}\/) pure(impureFunctionCall(...)) - which
allows calling impure functions in any pure context,\end{qquote}

$\mathrm{pure}$ accepts only one argument, although it can accept a
list of arguments by the syntax definition.

\subsubsection*{Predefined operators}

See Section~\ref{sec:predefined-operators}.

\subsubsection*{Result types}

A result type of an expression can be determined by subexpressions,
like C~Language.

%%%%%%%%%%%%%%%%

\subsection{Functions}

Arguments to a function are simple_types, records (including
connectors, operator_records) and arrays of them, and functions.

\begin{qquote} (\specref{12.2}\/) $\bullet$ A function may only
contain components of the restricted classes type, record, operator
record, and function; i.e. no model or block components.\end{qquote}

\subsubsection*{(?) No results}

An output of an empty array is used to return no values.  See an
example in \specref{9.4.3.1}.

\begin{quote}
\begin{lstlisting}[aboveskip=-\baselineskip]
function equalityConstraint
    input AC_Angle theta1;
    input AC_Angle theta2;
    output Real residue[0] "No constraints"
    algorithm
\end{lstlisting}
\end{quote}

%%%%%%%%%%%%%%%%

\subsection{Equations}

\subsubsection*{if-equation}

Each branch of $\mathbf{if}$ is tested sequentially, and the first
true branch is taken.

\subsubsection*{for-equation}

A range of $\mathbf{for}$ is a vector expression (real or integer)
that is a translation-time constant, or types of Boolean or an
enumeration.  See Section~\ref{sec:for-iterator}.

%%%%%%%%%%%%%%%%

\subsection{Semantics remarks to grammar rules}

\paragraph {IDENT}

There is a single namespace for classes and variables, and {IDENT} can
refer to a class name or a variable name which is not distinguished in
the grammar rules.

{\it{}name\/} is a sequence of {IDENT}.  In {\it{}type-specifier},
IDENT in {\it{}name\/} refers to a class.  In
{\it{}element-modification}, IDENT refers to a variable.  In
{\it{}function-argument}, IDENT refers to a class, and a sequence of
them refers to a function.  Note that functions are classes.

In {\it{}enumeration-literal}, {\it{}declaration}, {\it{}for-indices},
{\it{}component-reference}, and {\it{}named-argument}, IDENT refers to
a variable.

In {\it{}external-function-call}, {\it{}class-specifier}, and
{\it{}der-class-specifier}, IDENT refers to a class.  In
{\it{}import-clause}, IDENT refers to a class, whose last IDENT can
refers to a class or a variable.

%%%%%%%%%%%%%%%%

\subsection{Additional rules}

\subsubsection*{(?) Steps of class name resolution}

Class name resolution in a class may not proceed to process component
classes at once.  It may stop at resolving classes of
import/extends-clauses, and postpone resolving component classes.

The following is code fragments in the MSL.

\begin{quote}
\begin{lstlisting}[aboveskip=-\baselineskip]
within;
package Modelica
  extends Modelica.Icons.Package;

within Modelica;
package Icons
  operator record TypeComplex = Complex;
\end{lstlisting}
\end{quote}

Name resolution may proceed as in the follow steps.  But, at the last
step, the base-classes are not ready to lookup in (.Modelica) yet.
%%%%
\begin{quote}
\begin{verbatim}
1. Process (.Modelica)
2. Resolve base-classes in (.Modelica)
    - Lookup base-class (Modelica.Icons.Package) in (.Modelica)
    - Lookup (Modelica) in (.)
    - Lookup (Icons) in (.Modelica)
3. Process (.Modelica.Icons) (* to find Package in it *)
4. Resolve component classes in (.Modelica.Icons)
    - Lookup component class (Complex) in (.Modelica.Icons)
    - Lookup (Complex) in (.Modelica) (* go up by enclosing class *)
    - Lookup (Complex) in base-classes of (.Modelica)
\end{verbatim}
\end{quote}

\subsubsection*{No non-each redeclarations}

It is not possible to apply distinct redeclarations to elements of an
array.  Splitting of a redeclaration is not defined, and there is no
way to drop each in the following.

\begin{quote}
\begin{lstlisting}[aboveskip=-\baselineskip]
A a[10] (each redeclare class X=B);
\end{lstlisting}
\end{quote}

This implies class redeclarations can be handled without determining
(folding constants, etc.) the array indices.

%%%%%%%%%%%%%%%%

\subsection{Memo}

\begin{itemize}

\item Classes needing special handling: connectors, states in
transitions, etc.

\item Compilers can ignore "package.order" files.  It mainly be used
to display classes in some order.

\item (?) Is there no problem on using a file name "A/B/package.mo"
for package descriptions (a partial support is that "package" is
usually not used as a class name).

\item Arrays are described in (\specref{4.5.1}\/).

\item Class $\mathit{Complex}$ is defined in the root, not in
$\mathit{.Modelica}$.

\item Modifiers to a class do not directly applies to an
extends-redeclaration but to a replaced class.

\item (?) $\mathbf{protected}$ inner/outer may be illegal.

\item (?) inner/outer compatibility of arrays.

\item Connections .branch, .root, .potentialRoot, .isRooted.

\end{itemize}

\begin{QFACT}\label{class-lookup} A class definition needs to be
obtained by following the hierarchy, because a class may be modified
by its enclosing classes (by replacing itself or its
parts). \end{QFACT}

\if 0
\item $\mathbf{each}$: ($\mathbf{each}$ in a redeclaration in modifiers be
dropped at application to an array variable).
\fi

\if 0
Prefix Rules [4.4.2.2 Prefix Rules].
* functions: prefixes define causality.
* models/blocks: input prefix defines the need of binding equations.
* connectors: input/ouput prefix defines the correspondence of connectors.
(?) Causality condition of functions is unclear.
\fi

%%%%%%%%%%%%%%%%%%%%%%%%%%%%%%%%%%%%%%%%%%%%%%%%%%%%%%%%%%%%%%%%

\section{Some noteworthy excerpts}

%%%%%%%%%%%%%%%%

\subsection{Terms with unnamed}

There are many uses of {\it{}unnamed\/} in the_specification.

\begin{quote}
\begin{description}

\item [unnamed enclosing class]
in {5.2 Enclosing Classes}

\item [unnamed root of the class tree]
in \specref{5.6.1.1}

\item [unnamed nodes in the instance tree]
in \specref{5.6.1.2}
(unnamed nodes are generated by extends clauses)

\item [unnamed root of the instance tree]
in \specref{5.6.2}

\item [unnamed top-level package]
in \specref{13.2.1.1}

\item [unnamed top-level scope]
in \specref{13.2.4.1}

\item [unnamed element of a class definition]
in \specrefx{}{Glossary}
As "extends clause: an unnamed element of a class definition"

\end{description}
\end{quote}

%%%%%%%%%%%%%%%%

\subsection{Quotes about class name lookups}

\def\partiallyflattenedclass{partially flattened class}
\def\temporarilyflattenedclass{temporarily flattened class}

\paragraph{Qualified package~names}

\begin{qquote}\label{quo:lookup-importing} (\specref{13.2.1}\/) Here
packagename is the fully qualified name of the imported package~...
(\specref{13.2.1.1}\/) ...~Lookup of the name of an imported package
or class, ..., deviates from the normal lexical lookup by starting the
lexical lookup of the first part of the name at the
top-level.\end{qquote}

\paragraph{Remaining parts lookups}

A lookup of the remaining parts of a name includes the elements from
the base_classes.  It is because the lookup is in the
\temporarilyflattenedclass.
%%%%
\begin{qquote}\label{quo:lookup-remaining} (\specref{5.3.2}, 4th-item) If
the identifier denotes a class, that class is temporarily flattened
...  The rest of the name ... is looked up among the declared named
elements of the temporary flattened class.  If the class does not
satisfy the requirements for a package, the lookup is restricted to
encapsulated elements only.\end{qquote}
%%%%
And, ... becuase the \temporarilyflattenedclass is usual flattening
with empty modifications.
%%%%
\begin{qquote} (\specref{5.3.2}, in the following comment) [The
temporary class flattening performed for composite names follow the
same rules ..., except that the environment is empty.~...]\end{qquote}
%%%%
And, the \partiallyflattenedclass includes the elements from the
base_classes.
%%%%
\begin{qquote}\label{quo:partially-flattened-class} (\specref{5.2}\/)
During flattening, the enclosing class of an element being flattened
is a partially flattened class.  [For example, this means that a
declaration can refer to a name inherited through an
extends-clause.]\end{qquote}

%%%%%%%%%%%%%%%%%%%%%%%%%%%%%%%%%%%%%%%%%%%%%%%%%%%%%%%%%%%%%%%%

\section{Miscellaneous implementation tips}

\subsection{Odd facts}

\begin{itemize}

\item It is said that the RML compiler of OpenModelica is written in
SML~'97 (Standard ML).  RML is Relational Meta-Language.  Refer to
OpenModelica for MetaModelica and RML.

\end{itemize}

\subsection{Code indentation (emacs setting)}

Code is indented by emacs sml-mode.el (6.9).  It is with minor
modifications to adjust spacing in the first mrule after
{\tt{}"case~x~of~pat~=>~({\char`\\}n} to align with other mrules.  And
similarly after {\tt{}"fun~f~=~({\char`\\}n"}.

%%%%
\begin{quote}
\small
\begin{lstlisting}[aboveskip=-\baselineskip]
--- sml-mode.el.org
+++ sml-mode.el
@@ -530,14 +530,20 @@
     (`(:after . "struct") 0)
     (`(:after . "=>") (if (smie-rule-hanging-p) 0 2))
     (`(:after . "in") (if (smie-rule-parent-p "local") 0))
-    (`(:after . "of") 3)
+    (`(:after . "of") 2)
-    (`(:after . ,(or `"(" `"{" `"[")) (if (not (smie-rule-hanging-p)) 2))
+    (`(:after . ,(or `"(" `"{" `"["))
+     (if (not (smie-rule-hanging-p))
+        2
+        (if (or (smie-rule-parent-p "|") (smie-rule-parent-p "of"))
+            (+ sml-indent-level -2)
+            (+ sml-indent-level 0))))
     (`(:after . "else") (if (smie-rule-hanging-p) 0)) ;; (:next "if" 0)
     (`(:after . ,(or `"|" `"d|" `";" `",")) (smie-rule-separator kind))
     (`(:after . "d=")
-     (if (and (smie-rule-parent-p "val") (smie-rule-next-p "fn")) -3))
+     (if (and (smie-rule-parent-p "val") (smie-rule-next-p "fn")) -3 2))
     (`(:before . "=>") (if (smie-rule-parent-p "fn") 3))
-    (`(:before . "of") 1)
+    (`(:before . "of") 2)
     ;; FIXME: pcase in Emacs<24.4 bumps into a bug if we do this:
     ;;(`(:before . ,(and `"|" (guard (smie-rule-prev-p "of")))) 1)
     (`(:before . "|") (if (smie-rule-prev-p "of") 1 (smie-rule-separator kind)))
\end{lstlisting}
\end{quote}
%%%%

%%%%%%%%%%%%%%%%%%%%%%%%%%%%%%%%%%%%%%%%%%%%%%%%%%%%%%%%%%%%%%%%

\section{Supplemental glossary}

%%%%%%%%%%%%%%%%

\subsection{Words specific to the_implementation}

Some words are coined for the_implementation or used with different
meaning.  The marker {$\mbox{}^\top$} indicates the word is with
stricter meaning.  The marker {$\mbox{}^\bot$} indicates the word with
different meaning.

\begin{description}

\item [analogical] is used to refer to a set of flow, effort, and
stream.

\item [array, scalar, vector, matrix].  Scalars are arrays with no
dimensions, vectors with one dimension, and matrices with two
dimensions.  There are no separate vector nor matrix types.

\item [component{$^\top$} {\rm{}and} element{$^\top$}] component
{$\subseteq$} element as a set in a class definition.  Components are
variable declarations.  Elements are both variable declarations and
class definitions.

\item [definition/declaration] are often distinguished to mean a class
definition and a variable declaration.  It adheres to the uses in the
syntax rules.  It is the same for redefinition/redeclaration.
Declaration may be used to refer to both.

\item [definition-body] is a set of elements of a definition that is
not short-class-definition.

\item [lexically enclosing class] is used as an enclosing class
without modifications and $\mathbf{encapsulated}$ considered.

\item [instance/state/variable] are used interchangeably.

\item [internal_connection] is an aspect of a connector, which refers
to a connection to an outside_connector.

\item [main_class] refers to a class which is not a base class.

\item [merging_modifiers] is to merge a part of prefixes and modifiers
in a redeclaration.  It is called inheriting in the_specification.

\item [predefined_type{$\mbox{}^\top$}] is simple_types and classes
defined in \specref{4.8.8}.  "Built-in_type" is synonymous and
avoided.

\item [present_class] is a class at attention.  It is called a current
class in the_specification.

\item [refining] is a modifier application, including one by a
short_class_definition.

\item [resolve] is determining the class to which a class name refers.

\item [simple_type{$\mbox{}^\top$}] is a set of {Real}, {Integer},
{Boolean}, {String}, and $\mathit{enumeration}$.  The_implementation
also includes types extending them.  (It does not includes der's).

(* They are called in many ways.  A BASIC DATA TYPE is Real, Integer,
Boolean and String.  A PREDEFINED VARIABLE TYPE is Real, Integer,
Boolean, and String.  A SIMPLE TYPE is Real, Integer, Boolean, String
and enumerations.  A TYPE is Real, Integer, Boolean, String,
enumerations, arrays of a type, and ones extending a type.  *)

\item [subject] is a name of a package or an instance.  It is used as
a key to an associated class definition.  For example, "tank.level" is
a subject of a state, ".Modelica.Fluid" is a subject of a package.

\item [$\in$] indicates an element is in a class or in a class of a
state variable.

%%%%%%%%%%%%%%%%

\item [attributes] are parameters defined in predefined_types (Real,
Integer, Boolean, and String) and enumerations.

\item [base-prefix] = input or output (see syntax rules).

\item [elemental redeclaration] is a redeclaration that appears in
class elements, in contrast to in modifiers.

\item [environment] (in \specref{7.1}\/) = {modification environment}.

\item [instance] (in \specref{5.4}\/) is a state, declared as a
variable of a certain class.

\item [instance~hierarchy] (in \specref{5.4}\/) is an embedding
relation of instances.  (Related words: instance~hierarchy,
instance~scope).  A hierarchy is created by an instance declaration in
a class definition.  It is distinct from lexically enclosings.

\item [instance~scope] (in~\specref{5.6.1.2}, \specref{5.3.1}\/)
Instance~scope is a scope excluding enclosing classes.  It consists of
declared elements, declared elements from base_classes, and
imported_names.

\item [instance~tree] = instance~hierarchy.

\item [(instance) model hierarchy{$\mbox{}^\bot$}]
(in~\specref{5.4}\/) = instance~hierarchy.

\item [instantiation{$^\bot$}] (in \specref{5.6.1}\/) refers to
applying modifications.  It makes a {partially instantiated} element
be fully instantiated.

\item [(instantiation) partial instantiation] (in \specref{5.6.1.4}\/)
is an intermediate state of flattening a class where modifiers are
associated (but not applied) to an element but redeclarations are
applied.  This state is necessary to enable redeclarations to be
applied before handling extends-clauses.

\begin{qquote} (\specref{5.6.1.4}\/) 2.1. ...~(the partially
instantiated elements have correct name allowing lookup)\end{qquote}

\item [(instantiation) partial instantiation] (in \specref{5.6.1.4}\/)
is to make a pair of a class element and a {modification environment},
where redeclarations are applied.  A produced pair is called a
{partially instantiated element}.

\begin{qquote} (\specref{5.6.1.4}\/) ...~a partially instantiated
element (...) is comprised of a reference to the original element
(from the class tree) and the modifiers for that element.\end{qquote}

\item [modification~environment{$^+$}] (or {\bf{}flattening
environment}) is a set of modifiers attached to a {\it{}partially
instantiated element}.

\begin{qquote} (\specref{7.2.2}\/) The {modification environment}
contains arguments which modify elements of the class.\end{qquote}

\item [flattening~context] is a pair of a {modification environment}
and a set of {enclosing classes} (\specref{5.1}\/).

\item [(flattening) partial flattening{$^+$}] = {\it{}partial
instantiation}, as instantiation is a part of flattening.

\item [(flattening) partially flattened enclosing class] is simply
{\it{}partially flattened\/} + {\it{}enclosing class\/} (See
\specref{7.1}\/).

\item [(flattening) partially flattened enclosing class]
From the quote, it means the extends-clauses are resolved.  Thus, the
names in the base_classes can be looked up.

\begin{qquote} (\specref{5.2}\/) During flattening, the enclosing
class of an element being flattened is a partially flattened
class. [For example, this means that a declaration can refer to a name
inherited through an extends-clause.]\end{qquote}

\item [(flattening) temporary flattening] (in \specref{5.3.2}\/) is a
flattening with an empty modification environmemnt.

\item [(flattening) modification environmemnt] ?.

\item [simple~name {\rm{}and} composite~name].

\item [built-in~classes] = (Synonyms) predefined_classes.

\item [import~name] (in \specref{5.3.1}\/) is an element imported by a
qualified import.

\begin{qquote} (\specref{5.1}\/) Flattening is made in a context which
consists of a {modification environment} (Section 7.2.2) and an
ordered set of enclosing classes.\end{qquote}

\item [temporary flattened class] is a flattened class with an empty
{modification environment}.  (* the following part is uncertain *) It
includes the declared_elements, elements from the extends-clauses, and
the elements from the import-clauses, but not the elements from the
enclosing classes.

\begin{qquote} (\specref{5.3.3}\/) ...~the class A is temporarily
flattened with an empty environment (i.e. no modifiers, see Section
7.2.2)~...\end{qquote}

\item [class tree] (in \specref{5.6.1.1}\/) is class definitions,
hierarchically organized.

\item [local class/component] (in \specref{5.6.1.4}\/) It is a
class/variable defined/declared in elements, not including inherited
classes.  It, probably, excludes imported, base, and enclosing
classes.

\item [lookup of base name] for {extends}.

\begin{qquote} (\specref{7.1}\/) The name of the base_class is looked up
in the partially flattened enclosing class.\end{qquote}

The sentence {\it{}The found base_class is flattened with~...\/} has
the recurrence of the word flattened as {\it{}arguments of all
enclosing class environments that match names in the flattened
base_class\/} in the following itemization.

\item [ordering of the merging rules]

\item [redeclaration] A redeclaration apprears in three ways
(\specref{7.3}\/): $\bullet$\,in modifiers; $\bullet$\,in class
definition elements; $\bullet$\,in extends-redeclaration.

For an element-redeclaration, a class/variable must be inherited.  The
modifiers on the extends-clause of the base_class are ignored
(\specref{7.3}\/).

\item [simple~types{$^+$}] are boolean, integer, enumeration, real,
string, Clock, and {\it{}externally defined object}.  (The definition
in glossary lacks the last two).

\item [specialized class] (\specref{4.6}\/) lists a list of classes
with specific kinds (e.g, $\mathbf{model}$, $\mathbf{block}$).

\item [unnamed top-level package] = ?

\end{description}

%%%%%%%%%%%%%%%%%%%%%%%%%%%%%%%%%%%%%%%%%%%%%%%%%%%%%%%%%%%%%%%%

\section{Issues need to be checked in MSL}

\subsection{Missing each}

An each seems missing for p, at line 850, Fluid/Interfaces.mo in
MSL~3.2.3:
%%%%
\begin{quote}
\begin{lstlisting}[aboveskip=-\baselineskip]
partial model PartialDistributedVolume
  ......
  Medium.BaseProperties[n] mediums(
    each preferredMediumStates=true,
    p(start=ps_start),
    each h(start=h_start),
    each T(start=T_start),
    each Xi(start=X_start[1:Medium.nXi]));
\end{lstlisting}
\end{quote}

\subsection{Mismatch of final statuses}

Final statuses do not match in a redeclaration in Fluid/Vessels:
%%%%
\begin{quote}
\begin{lstlisting}[aboveskip=-\baselineskip]
  package BaseClasses
      partial model PartialLumpedVessel
        replaceable model HeatTransfer =
            Modelica.Fluid.Vessels.BaseClasses.HeatTransfer.IdealHeatTransfer
          ...;

        HeatTransfer heatTransfer(
          redeclare final package Medium = Medium,
          ...);

    partial model PartialVesselHeatTransfer
      extends Modelica.Fluid.Interfaces.PartialHeatTransfer;
      ...

    model IdealHeatTransfer
      extends PartialVesselHeatTransfer;
      ...
\end{lstlisting}
\end{quote}
%%%%
and in Fluid/Interfaces.mo:
%%%%
\begin{quote}
\begin{lstlisting}[aboveskip=-\baselineskip]
  partial model PartialHeatTransfer
    replaceable package Medium=Modelica.Media.Interfaces.PartialMedium
    ...
\end{lstlisting}
\end{quote}

%%%%%%%%%%%%%%%%%%%%%%%%%%%%%%%%%%%%%%%%%%%%%%%%%%%%%%%%%%%%%%%%

\bibliographystyle{plain}
\bibliography{notes}

\stepcounter{section}
\addcontentsline{toc}{section}{\numberline{\arabic{section}}References}

%%%%%%%%%%%%%%%%%%%%%%%%%%%%%%%%%%%%%%%%%%%%%%%%%%%%%%%%%%%%%%%%

\appendix
%%\appendices

%%%%%%%%%%%%%%%%%%%%%%%%%%%%%%%%%%%%%%%%%%%%%%%%%%%%%%%%%%%%%%%%

\section{List of keywords and predefined names}

%% KEYWORDS AND BUILTINS

This appendix summarizes some list of keywords and predefined names,
etc.

%%%%%%%%%%%%%%%%

\subsection{Modelica keywords}

%% (Make quoted words separated; use ~ for usual spacing).
\def\widespacing{\spaceskip=2em%
\fussy\hyphenpenalty=10000\exhyphenpenalty=10000%
\let~=\enspace}
%%\sloppy\fussy\linepenalty=0\tolerance=0%

(\specref{2.3.3}\/)

\subsubsection*{Syntactic keywords}

\begin{quote}\widespacing
algorithm
and
annotation
block
break
class
connect
connector
constant
constrainedby
der
discrete
each
else
elseif
elsewhen
encapsulated
end
enumeration
equation
expandable
extends
external
false
final
flow
for
function
if
import
impure
in
initial
inner
input
loop
model
not
operator
or
outer
output
package
parameter
partial
protected
public
pure
record
redeclare
replaceable
return
stream
then
true
type
when
while
within
\end{quote}

\subsubsection*{Reserved type names}

\begin{quote}\widespacing
Real
Integer
Boolean
String
\end{quote}

\noindent
In addition to the syntactic keywords above, the type names are
reserved and cannot be redefined (See Quote~\ref{quo:reserved-words}).
These names have overloaded meanings.

(*CHECK*) Check the status of "constructor".

%%%%%%%%%%%%%%%%

\subsection{Predefined_classes}

(\specref{4.8}\/)

\subsubsection*{Classes}

\begin{quote}
\spaceskip=2em
\sloppy\hyphenpenalty=10000\exhyphenpenalty=10000

Real
Integer
Boolean
String
{\it{}enumeration}
Clock
StateSelect
ExternalObject
AssertionLevel
Connections

\end{quote}

\noindent $\mathit{enumeration}$ means names of any enumeration.  All
enumerations are members of simle_types.  It may be referred to as
{\it{}EnumTypeName\/} in the_specification.

\subsubsection*{Attributes}

\begin{quote}\widespacing

start
fixed
nominal
unbounded

\end{quote}

%%%%%%%%%%%%%%%%

\subsection{Predefined variable}

\subsubsection*{Time}

(\specref{3.6.7}\/)

\begin{quote}\widespacing
time
\end{quote}

\noindent $\mathit{time}$ is defined in the unnamed root.

%%%%%%%%%%%%%%%%

\subsection{Operators with special syntax}

\subsubsection*{Arithmetic operators (on scalars or arrays of Real, Integer)}

(\specref{10.6}\/)

\begin{quote}
\spaceskip=2em
\begin{verbatim}
+ - .+ .- * / ^ .* ./ .^ = :=
\end{verbatim}
\end{quote}

\noindent The descriptions span multiple sections:
\specref{10.6.2},
\specref{10.6.3},
\specref{10.6.4},
\specref{10.6.5},
\specref{10.6.6},
\specref{10.6.7},
\specref{10.6.8}.

\subsubsection*{Relational operators (on scalars of Real, Integer, Boolean, String)}

(\specref{10.6.10}\/)

\begin{quote}
\spaceskip=2em
\begin{verbatim}
== <> < <= > >=
\end{verbatim}
\end{quote}

\subsubsection*{Boolean operators (on scalars or arrays of Boolean)}

(\specref{10.6.11}\/)

\begin{quote}
\spaceskip=2em
\begin{verbatim}
and or not
\end{verbatim}
\end{quote}

\subsubsection*{Class constructors}

\begin{quote}
\spaceskip=2em
\begin{verbatim}
0 constructor
\end{verbatim}
\end{quote}

\noindent These are used in defining constructors.

\subsubsection*{Array constructor}
(\specref{10.4}\/)

\begin{quote}
\begin{verbatim}
{...}
\end{verbatim}
\end{quote}

\noindent This can be written as $\mathrm{array}\,(...)$.

\subsubsection*{Matrix constructor}
(\specref{10.4.2}\/)

\begin{quote}
\begin{verbatim}
[...]
\end{verbatim}
\end{quote}

\noindent "," is a column separator, and ";" is a row separator.  This
is a restricted notation of $\mathrm{cat}\,(...)$ for two dimensional
cases.

\subsubsection*{String literal}

\begin{quote}
\begin{verbatim}
"..."
\end{verbatim}
\end{quote}

\subsubsection*{String operators}

\begin{quote}
\begin{verbatim}
+
\end{verbatim}
\end{quote}

\subsubsection*{Array indexing operator}
(\specref{10.5}\/)

\begin{quote}
\begin{verbatim}
_ [...]
\end{verbatim}
\end{quote}

\subsubsection*{Record access operator}

\begin{quote}
\begin{verbatim}
.
\end{verbatim}
\end{quote}

\subsubsection*{Function call}

\begin{quote}
\begin{verbatim}
_ (...)
\end{verbatim}
\end{quote}

\subsubsection*{Array range}
(\specref{10.4.3}\/)

%%({3.6.4 Array Range Operator} is just a pointer to {10.4.3})

\begin{quote}
\spaceskip=2em
\begin{verbatim}
_:_ _:_:_
\end{verbatim}
\end{quote}

\noindent See Section~\ref{sec:array-ranges}.

\subsubsection*{Tuples}
(\specref{8.3.1}, \specref{11.2.1.1}\/)

\begin{quote}
\spaceskip=2em
\begin{verbatim}
(...)
\end{verbatim}
\end{quote}

\noindent Tuples are comma separated expressions.  Tuples are only
used in equalities and assignments for functions returning multiple
values.

%%%%%%%%%%%%%%%%

\subsection{Predefined functions}
(\specref{3.7}\/)

Predefined functions are defined in the unnamed root, and it can be
accessed with a preceding dot $.\mathit{abs}\,(v)$ for
$\mathit{abs}\,(v)$.  Array constructor admits an iterator syntax
($\mathrm{array}$).  Reductions admit a reduction syntax in addition
to a function call syntax ($\mathrm{min}$, $\mathrm{max}$,
$\mathrm{sum}$, and $\mathrm{product}$).  $\mathrm{der}$ and
$\mathrm{pure}$ are keywords but they can also be used by a function
call syntax.

\subsubsection*{Numeric functions and conversion functions}
(\specref{3.7.1}\/)

\begin{quote}\widespacing

abs(v)
sign(v)
sqrt(v)
Integer(e)
{\it{}enumeration}(i)
String(...)

\end{quote}

\subsubsection*{Event triggering mathematical functions}
(\specref{3.7.1.1}\/)

\begin{quote}\widespacing

div(x,~y)
mod(x,~y)
rem(x,~y)
ceil(x)
floor(x)
integer(x)

\end{quote}

\subsubsection*{Predefined mathematical functions}
(\specref{3.7.1.2}\/)

\begin{quote}\widespacing

sin(x)
cos(x)
tan(x)
asin(x)
acos(x)
atan(x)
atan2(y,~x)
sinh(x)
cosh(x)
tanh(x)
exp(x)
log(x)
log10(x)

\end{quote}

\subsubsection*{Derivative and special purpose operators}
(\specref{3.7.2}\/)

\begin{quote}\widespacing

der(expr)
delay(expr,~delayTime,~delayMax)
delay(expr,~delayTime)
cardinality(c)
homotopy(actual,~simplified)
semiLinear(x,~positiveSlope,~negativeSlope)
inStream(v)
actualStream(v)
spatialDistribution(in0,~in1,~x,~positiveVelocity,~initialPoints,~initialValues)
getInstanceName()

\end{quote}

\noindent $\mathrm{der}\,(...)$ accepts only one argument, although it
can accept a list of arguments by the syntax definition.

inStream(v) and actualStream(v) introduce a new state variable and
equations related to a given steam variable $v$.

\subsubsection*{Event-related operators}
(\specref{3.7.3}\/)

\begin{quote}\widespacing

initial()
terminal()
noEvent(expr)
smooth(p,~expr)
sample(start,~interval)
pre(y)
edge(b)
change(v)
reinit(x,~expr)

\end{quote}

\subsubsection*{Assert}
(\specref{8.3.7}\/)

\begin{quote}\widespacing

assert(condition,~message,~level)

\end{quote}

\subsubsection*{Terminate}
(\specref{8.3.8}\/)

\begin{quote}\widespacing

terminate(s)

\end{quote}

\subsubsection*{Array dimension and size functions}
(\specref{10.3.1}\/)

\begin{quote}\widespacing

ndims(A)
size(A,~i)
size(A)

\end{quote}

\noindent See~\ref{sec:array-size}.

\subsubsection*{Dimensionality conversion functions}
(\specref{10.3.2}\/)

\begin{quote}\widespacing

scalar(A)
vector(A)
matrix(A)

\end{quote}

\subsubsection*{Array constructor}
(\specref{10.4.1}\/)

\begin{quote}\widespacing
array(...)
\end{quote}

\noindent This can be written as $\{...\}$.

\subsubsection*{Specialized array constructor functions}
(\specref{10.3.3}\/)

\begin{quote}\widespacing
identity(n)
diagonal(v)
zeros(n1,~n2,~n3,~...)
ones(n1,~n2,~n3,~...)\\
fill(e,~n1,~n2,~n3,~...)
linspace(x1,~x2,~n)
\end{quote}

\noindent See~\ref{sec:array-size}.

\subsubsection*{Array concatenation}
(\specref{10.4.2}\/)

\begin{quote}\widespacing
cat(k,~A,~B,~C,~...)
\end{quote}

\noindent This can be written as $[...]$ in the two dimensional
case.

%% \subsubsection*{Predefined array functions}
%% (\specref{10.3})
%% \begin{quote}
%% \spaceskip=2em
%% \sloppy\hyphenpenalty=10000\exhyphenpenalty=10000
%% \let~=\enspace
%% promote(A,n)
%% \end{quote}

\subsubsection*{Matrix and vector algebra functions}
(\specref{10.3.5}\/)

\begin{quote}\widespacing
transpose(A)
outerProduct(v1,~v2)
symmetric(A)
cross(x,~y)
skew(x)
\end{quote}

\subsubsection*{Reduction functions and operators}
(\specref{10.3.4}\/)

\begin{quote}\widespacing

min(A)
min(x,~y)
min(e(i,~...,~j)~for~i~in~u,~...,~j~in~v)
max(A)
max(x,~y)
max(e(i,~...,~j)~for~i~in~u,~...,~j~in~v)
sum(A)
sum(e(i,~...,~j)~for~i~in~u,~...,~j~in~v)
product(A)
product(e(i,~...,~j)~for~i~in~u,~...,~j~in~v)

\end{quote}

\subsubsection*{pure}
(\specref{12.3}\/)

\begin{quote}\widespacing
pure(...)
\end{quote}

\noindent See Section~\ref{pure-expression}

%%%%%%%%%%%%%%%%

\subsection{Predefined elements}

\subsubsection*{Functions in types and records}

\begin{quote}\widespacing

equalityConstraint(x,~y)

\end{quote}

%%%%%%%%%%%%%%%%

\subsection{Synchronous language}

\subsubsection*{Clock class}

\begin{quote}
class Clock;
\end{quote}

\subsubsection*{Discrete states}
(\specref{16.4}\/)

\begin{quote}\widespacing

previous(u)

\end{quote}

\subsubsection*{Base-clock conversion operators}
(\specref{16.5.1}\/)

\begin{quote}\widespacing

sample(u,~c)
hold(u)

\end{quote}

\subsubsection*{Sub-clock conversion operators}
(\specref{16.5.2}\/)

\begin{quote}\widespacing

subSample(u,~factor)
superSample(u,~factor)
shiftSample(u,~shiftCounter,~resolution)
backSample(u,~backCounter,~resolution)
noClock(u)

\end{quote}

\subsubsection*{Other operators}
(\specref{16.10}\/)

\begin{quote}\widespacing

firstTick(u)
interval(u)

\end{quote}

%%%%%%%%%%%%%%%%

\subsection{State machines}

\subsubsection*{Transitions}
(\specref{17.1}\/)

\begin{quote}\widespacing

transition(from,~to,~condition,~immediate,~reset,~synchronize,~priority)
initialState(state)
activeState(state)
ticksInState()
timeInState()

\end{quote}

%%%%%%%%%%%%%%%%%%%%%%%%%%%%%%%%%%%%%%%%%%%%%%%%%%%%%%%%%%%%%%%%

\section{Mandatory annotations (maybe)}

\def\quotespace{\spaceskip=2em%
\sloppy\hyphenpenalty=10000\exhyphenpenalty=10000}

There are some annotations predefined (chapter 18): annotations for
graphical representation, annotations for code generation (18.3),
annotations for simulation experiments (18.4), annotations for
functions (12.7, 12.8 and 12.9).  Some of them are likely mandatory to
translators.

\subsubsection*{Derivative and inverses}
(\specref{12.7.1}, \specref{12.8}\/)

\begin{quote}\spaceskip=2em
derivative
inverse
\end{quote}

These directives are placed in function body elements.

\subsubsection*{(Redeclaration choices)}
(\specref{7.3.4}\/)

\begin{quote}\spaceskip=2em
choices
choice
\end{quote}

(?) The choices directive is associated to replaceables or modifiers,
and is placed at replacable and non-replacable elements by
short_class_definitons or declarations.

\subsubsection*{Fortran interface}
(\specref{12.9.1.2}\/)

\begin{quote}\spaceskip=2em
arrayLayout
\end{quote}

\subsubsection*{External libraries}
(\specref{12.9.4}\/)

\begin{quote}\spaceskip=2em
Library
Include
IncludeDirectory
LibraryDirectory
\end{quote}

\subsubsection*{Code generation}
(\specref{18.3}\/)

\begin{quote}\spaceskip=2em
Evaluate
HideResult
Inline
LateInline
GenerateEvents
smoothOrder
\end{quote}

$\mathit{Evaluate}$ and $\mathit{HideResult}$ affect component
declarations.  They are attached to component declarations, class
definitions, and a base class of class definitions.
$\mathit{Evaluate}$ is ignored for non-parameter components.

$\mathit{Inline}$, $\mathit{LateInline}$, $\mathit{GenerateEvents}$,
and $\mathit{smoothOrder}$ are placed in elements of function
definitions.

$\mathit{smoothOrder}=n$ is placed in algorithm sections of functions.
It takes optional argument $\mathit{normallyConstant}$.

\subsubsection*{Simulation setting}
(\specref{18.4}\/)

\begin{quote}\spaceskip=2em
experiment
StartTime
StopTime
Interval
Toleranc
\end{quote}

(?) $\mathit{experiment}$ is placed in class body elements.

\subsubsection*{Singleton instances}
(\specref{18.5}\/)

\begin{quote}\spaceskip=2em
singleInstance
\end{quote}

(?) $\mathit{singleInstance}$ is palced in class body elements.

\subsubsection*{Diagnostics messages}
(\specref{18.7}\/)

\begin{quote}\spaceskip=2em
missingInnerMessage
obsolete
unassignedMessage
\end{quote}

$\mathit{obsolete}$ applies to classes.

\subsubsection*{Version setting}
(\specref{18.8.2}\/)

\begin{quote}\spaceskip=2em
version
conversion
uses
\end{quote}

%%%%%%%%%%%%%%%%%%%%%%%%%%%%%%%%%%%%%%%%%%%%%%%%%%%%%%%%%%%%%%%%

\end{document}

%%%%%%%%%%%%%%%%%%%%%%%%%%%%%%%%%%%%%%%%%%%%%%%%%%%%%%%%%%%%%%%%

================================================================

%% README.md Section

# Baby-Modelica 3.4.0 (2020-01-29)

Copyright (C) 2018-2021 RIKEN R-CCS

__Baby-Modelica comes with ABSOLUTELY NO WARRANTY.__

Baby-Modelica is a simple frontend (parser+flattener) for the Modelica
language specification 3.4.  It parses the source code then performs
simple syntactic transformations, and optionally dumps a flat model.
Flattening a model is an essential part of Modelica, and it resolves
all the features as an object-oriented language.  Flattening is
indispensable even for simple tools, because parameters in a model are
substituted multiple times by nested parameterization.  A generated
flat model is hopefully to be re-readable by Modelica compilers.

Current status is a pre-zero version, and we are working towards
version zero.  The version number of Baby-Modelica is the last digits,
appended to the version number of the Modelica language specification.

Baby-Modelica is written in SML'97 (Standard ML) and developed mainly
with Poly/ML (on SUNOS 5.11/amd64), but it may work with SML/NJ or
MLton.  It uses a modified BYACC parser generator (not ml-yacc), which
needs be obtained separately.

See [https://www.modelica.org](https://www.modelica.org) for
information of Modelica.

If someone might be unfamiliar with SML, see
[http://sml-family.org](http://sml-family.org).  Poly/ML is at
[https://www.polyml.org](https://www.polyml.org), SML/NJ is at
[https://smlnj.org](https://smlnj.org), and MLton is at
[https://mlton.org](https://mlton.org).

----

## Source code files and directories

See frontend/notes.pdf first, then start with frontend/bbm.sml.
* [frontend/notes.pdf](frontend/notes.pdf): random implementation notes
* frontend/bbm.sml: a list of the source code files

The toplevel directories are as follows.
* [frontend](frontend): the source code
* [silly-models](silly-models): trivial code to check the specification
* [code-examples](code-examples): code snippets from the specification

----

## A simple test run with Poly/ML or SML/NJ

Set the environment variable "MODELICAPATH" to the library paths of
the Modelica Standard Library.  The paths point to the directory
containing such as "Modelica 3.2.3".  Use the MSL-3.2.3 (2019-01-23)
for testing.

Test with the following commands in the "frontend" directory.
"bbm.sml" loads the needed source files.  Flatdumper dumps a flat
model in "x.mo".

```
use "bbm.sml" ;
builder.xreset () ;
builder.xbuild "Modelica.Fluid.Examples.HeatingSystem" ;
connector.xbind () ;
connector.xconnect () ;
flatdumper.xdump () ;
valOf (classtree.xfetch1 "tank") ;
valOf (classtree.xfetch1 "wall.G") ;
valOf (classtree.xfetch1 "") ;
valOf (classtree.xfetch1 ".Modelica") ;
......
```

## Verbosity settings in PolyML

* Printer setting:
```
PolyML.print_depth (1000) ;
```

* Backtrace setting:
```
PolyML.Compiler.debug := true ;
open PolyML.Debug ;
breakEx Match ;
```

(Reloading files (to recompile) is needed after enabling backtracing).

## Verbosity settings in SML/NJ

* Printer setting:
```
Control.Print.printDepth := 100 ;
Control.Print.printLength := 10000 ;
```

* Backtrace setting:
```
CM.make "$smlnj-tdp/back-trace.cm" ;
SMLofNJ.Internals.TDP.mode := true ;
```

(Reloading files (to recompile) is needed after enabling backtracing).

----

Baby-Modelica is distributed under the BSD 2-Clause License.

================================================================

%% Local Variables:
%% indent-tabs-mode: nil
%% paragraph-start: "\f\\|[ \t]*$\\|%%%%$"
%% paragraph-separate: "[ \t\f]*$\\|%%%%$"
%% End:
